\chapter{Formation of the Sanctuary Doctrine}
\label{ch:formation}

\newthought{Pioneer Seventh-day Adventists} inherited their identification of the year
1844 as the terminus of the 2300 ``days'' foretold in the KJV of Daniel 8:14
from William Miller. Formerly an avowed skeptic, he was converted in 1816
and eventually became a Baptist lay preacher. He devoted his first two years 
as a born-again Christian to a diligent study of the Bible, which eventually
came to a focus on Daniel 8:14 and the conclusion that it foretold the
second coming of Christ ``about the year 1843.''

According to the Seventh-day Adventist Encyclopedia Miller ``repeatedly
declared that his prophetic views were not new'', but insisted that he came
to his conclusions exclusively through his own study of the Bible and
reference to a concordance. In volume 4 of his Prophetic Faith of Our
Fathers Le Roy Edwin Froom notes that Miller was by no means the
``originator'' of the idea that the 2300 ``days'' were prophetic years ending
about 1843, and that it is ``a simple historical fact that the origin of the
view of the 2,300 years as ending at that time, and its wide circulation,
was wholly prior to and independent of William Miller.''\sidenote{\bibentry{1}, page 403}

By what process did Miller, this formidable array of Bible students, and
pioneer Adventists arrive at 1843/44 as the terminus of the 2300 ``days'' of
Daniel 8:14? Relying on the 1611 King James translation of the Bible (the
only one then available), they
\begin{enumerate}
    \item identified its ``sanctuary'' as the church on earth
    \item accepted the KJV interpretation of erev boquer (literally, ``evening
        morning'') as ``days''
    \item adopted the ``day-for-a-year'' principle in Bible prophecy and thus
        construed the 2300 ``days'' as prophetic years
    \item took the seventy ``weeks'' of Daniel 9:24-27 as the first segment of
        these 2300 years
    \item identified the cessation of sacrifice and offering for the last half
        of the seventieth of the seventy ``weeks'' (verse 27) as referring to
        Jesus' crucifixion\footnote{Matt. 57:21 At that moment the curtain of
        the temple was torn in two from top to bottom. The earth shook, the
        rocks split.}
    \item figuring back from the crucifixion, they identified the decree of the
        Persian king Artaxerxes Longimanus in his seventh year (Ezra 7) as that
        alluded to in Daniel 9:25, thus locating the commencement of the 2300
        years in 457 B.C.
    \item with 457 B.C. as their starting point, terminated them ``about the
        year 1843''
    \item adopted the KJV interpretation of nitsdaq (literally, ``set right''
        or ``restored'') as ``cleansed'', and 
    \item concluded that the cleansing of the sanctuary of Daniel 8:14
        meant the cleansing of the church on earth (and thus the earth
        itself) by fire at the second coming of Christ.
\end{enumerate}

When the great disappointment of October 22, 1844 proved conclusively that
Miller's identification of the ``sanctuary'' in Daniel 8:14 as the church on
earth and the nature of its cleansing as by fire at the second coming of
Christ\sidenote{2 Peter 3:7--12 By the same word the present heavens and earth
are reserved for fire, being kept for the day of judgment and destruction of the
ungodly. But do not forget this one thing, dear friends: With the Lord a day is like a
thousand years, and a thousand years are like a day. The Lord is not slow in
keeping his promise, as some understand slowness. Instead he is patient with
you, not wanting anyone to perish, but everyone to come to repentance.
But the day of the Lord will come like a thief. The heavens will disappear
with a roar; the elements will be destroyed by fire, and the earth and everything
done in it will be laid bare.
Since everything will be destroyed in this way, what kind of people ought
you to be? You ought to live holy and godly lives as you look forward to the
day of God and speed its coming. That day will bring about the destruction of
the heavens by fire, and the elements will melt in the heat.},
 were in error, pioneer Adventists re-identified the ``sanctuary'' of
verse 14 as that of the Book of Hebrews in heaven\sidenote{Heb. 8:2 and who
serves in the sanctuary, the true tabernacle set up by the Lord, not by a mere
human being.}, and its cleansing as the heavenly counterpart of the cleansing
of the ancient sanctuary on the Day of Atonement\sidenote{Lev. 16}.

Retaining, however, the presumed validity of October 22, 1844 as the
fulfillment of Daniel 8:14 and the concept that it implied the soon return
of their Lord, the disappointed Adventist pioneers assumed that human
probation had indeed closed on that fateful day, and that only those who at 
that time awaited His return were eligible for eternal life. They referred
to this concept as ``the shut door'' in the parable of the Ten
Virgins\sidenote{Matt. 25:1--13}.
 They
soon mated the ``shut door'' theory to the idea that the sanctuary of Daniel
8:14 was the sanctuary in heaven, of the book of Hebrews, that the ``shut
door'' was the ``door'' between its holy and most holy apartments, that on
October 22 Christ had closed His ministry in the holy place and entered upon
His high priestly ministry in its most holy place, and referred to His
ministry there as an ``investigative judgment''.

For several years the ``little flock'' of pioneer Seventh-day Adventists
``scattered abroad'' believed that the investigative judgment phase of
Christ's ministry would be very brief (not more than five years or so at the
most\sidenote{\bibentry{7}, page 58}),
 following which He would immediately return to earth. The eventual
accession of new, non-1844, members to the ``little flock'' proved to be
convincing evidence that the door of mercy remained open, and by the early
1850's they abandoned the ``shut door'' aspect of the sanctuary-in-heaven
interpretation of Daniel 8:14.

This completed the traditional Adventist interpretation of Daniel 8:14, the
sanctuary, and the investigative judgment, which was thereafter commonly
referred to as ``the sanctuary doctrine'' set forth in every statement of
beliefs, most recently as article 23 of the 27 Fundamental Beliefs adopted
at the 1980 session of the General Conference in New Orleans.
