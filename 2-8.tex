\chapter{``Rightly Explaining'' Daniel 8:14}
\label{ch:explain}

\newthought{The first imperative} for comprehending the prophecies of Daniel in the sense
Inspiration intended is an objective frame of mind divested of every
personal, subjective, modern presupposition with respect to their import.

The second imperative is to identify the circumstances set forth in Daniel 1
to 6 and 9:1-23, which provide the historical background within which
Inspiration set its five prophetic passages and from which it intended
Daniel and his intended readers to understand them. Accordingly, in order to
understand those passages as Inspiration intended them to be understood we
must do so with that historical perspective in our minds, and from the same
perspective of salvation history as Daniel and his intended readers did. Any
interpretation that ignores or controverts that historical perspective and /
or the salvation history perspective of their time is automatically suspect

The first six chapters of the Book of Daniel recount the exile of Daniel and
his compatriots to Babylon ``in the third year of the reign of Jehoiakim of
Judah,'' which is dated to 606/5 B.C., and their experiences during the
seventy years of exile foretold by Jeremiah in chapter 29:1-14. According to
Daniel 9:1, in ``the first year of Darius'' (which is dated to 537/6 B.C. by
Jewish inclusive reckoning), Daniel had been in exile for exactly seventy
years. But as yet there was no visible evidence that release from exile was
imminent. Accordingly, Daniel prayed the importunate prayer for release from
exile and for restoration recorded in chapter 9:4-19. 
and imposes an alien, uninspired interpretation on those prophecies. 

While Daniel was still praying the angel Gabriel 
reappeared\sidenote{\nameref{ch:daniel9}:23, cf. \nameref{ch:daniel8}:16} and said, ``I
have now come out to give you wisdom and understanding. At the beginning of
your supplications a word went out [obviously in heaven], and I have come to
declare it, for you are greatly beloved. So consider the word and understand
the vision.'' Gabriel thereupon repeats that ``word'' verbatim (verse 24), as
he had promised, and proceeds to explain it in verses 25 to 27. 

It is of crucial importance to note that Gabriel explicitly identifies the
``word'' that ``went out to restore and build Jerusalem'' at the commencement of
the seventy weeks of years as ``the word'' that ``went out''--in heaven--while
Daniel was praying.\sidenote{\nameref{ch:daniel9}:21-23} That 
``word''\sidenote{\nameref{ch:daniel9}:24} was obviously one that only God Himself 
(and not an earthly monarch) could possibly have issued! On the authority of
no less a person than the angel Gabriel, the ``seventy weeks'' of years thus
began in 537 B.C., not eighty years later in 457 B.C.!

Gabriel's explanation of that ``word'' in verses 25--27 very briefly sketched
the future of God's covenant people during the seventy weeks of years, and
its climax in the ruthless oppression of ``the prince who is to come'' during
the seventieth of the seventy ``weeks,'' which he had already foretold in
chapter 8:9--13 and explained in verses 19 to 
25.\sidenote{cf. \nameref{ch:daniel7}:24-25}

As already noted, Daniel 9:23-25 begins the seventy weeks of years at the
time the ``word'' was issued in heaven, in 537 B.C. In the same way,
contextual identification of the ``he'' of verse 27 identifies events of
history that mark their close in the seventieth of the seventy ``weeks.'' It
is universally accepted that the immediate antecedent of a personal pronoun
identifies the person to whom it refers unless the context unambiguously
specifies otherwise. Accordingly, verse 26 identifies the immediate
antecedent of the pronoun ``he'' in verse 27, who ``make[s] a strong covenant
with many'' for the seventieth of the seventy ``weeks'' and ``make[s] sacrifice
and offering cease'' during the last half of the ``week,'' as the evil ``prince
who is to come''--not the ``anointed prince'' of verses 25--26! 

Chapter 11:23 confirms the fact that his alias, the last king of the north,
does, indeed make such a covenant with people in ``alliance'' with him. Also,
his fate set forth in verse 27, ``the decreed end is poured out on the
desolator,'' is equivalent to the horn-king of chapter 8:25 being ``broken,
and not by human hands,'' and to the last king of the north in chapter 11 who
``come[s] to his end, with no one to help him.''\sidenote{\nameref{ch:daniel11}:45} 

Chapter 9:24--27 thus provides an exact but much more complete explanation of
chapter 8:13-14's question and answer about events between Daniel's time and
``the appointed time of the end'' ``many days from now'' when ``the vision of the
evenings and the mornings'' was to meet its 
fulfillment.\sidenote{\nameref{ch:daniel8}:17, 26} Isn't that exactly
what Gabriel said the audition of 9:24-27 was supposed to 
do?\sidenote{\nameref{ch:daniel9}:22-25} 

Such is Daniel's perspective of salvation history. In order to understand
chapters 8 and 9 as heaven intended them to be understood, we must imagine
ourselves in Daniel's historical circumstances and view them from his
perspective of salvation history in order to form an accurate understanding
of what was revealed to him.

\section{Daniel's Perspective of Salvation History}

Daniel's perspective of salvation history was a composite of the visions of
chapters 2 and 7, each with its explanation, and chapter 8 with
its three-fold explanation in chapters 8, 9, and 11-12. It consisted of a 
series of universal kingdoms\sidenote{\nameref{ch:daniel2}:37-40, \nameref{ch:daniel7}:3-7,
\nameref{ch:daniel8}:3-8, \nameref{ch:daniel11}:2-3} followed by a period of disintegration and
fragmentation,\sidenote{\nameref{ch:daniel2}:41-43, \nameref{ch:daniel7}:7-8, 17, 23,
\nameref{ch:daniel8}:8-9, \nameref{ch:daniel8}:8-9, \nameref{ch:daniel11}:4-5, 25-29,
40-43} which Gabriel told Daniel would be a ``troubled time''
(9:25)\sidenote{\nameref{ch:daniel9}:25}

At the ``appointed time of the end \ldots many days from now''--after sixty-nine
of the ``seventy weeks of years''\sidenote{\nameref{ch:daniel2}:44, \nameref{ch:daniel7}:28,
\nameref{ch:daniel8}:17, 19, 26, \nameref{ch:daniel9}:24, 27, \nameref{ch:daniel11}:35, 40}
--there would be an unprecedented ``time of
anguish'' for God's people in which they would be ``trampled,'' their power
shattered,\sidenote{\nameref{ch:daniel7}:21, 25, \nameref{ch:daniel8}:10, 13, 24-25,
\nameref{ch:daniel9}:26, \nameref{ch:daniel12}:1, 2, 7} their land and city 
devastated,\sidenote{\nameref{ch:daniel8}:9, \nameref{ch:daniel9}:36,
\nameref{ch:daniel11}:22, 24, 41} their loyalty and
faithfulness to God tested,\sidenote{\nameref{ch:daniel8}:11, 25, \nameref{ch:daniel11}:36} 
their covenant with Him and its prescribed
system of worship abolished,\sidenote{\nameref{ch:daniel7}:25, \nameref{ch:daniel8}:11-12,
\nameref{ch:daniel9}:26-27, \nameref{ch:daniel11}:31, \nameref{ch:daniel12}:11}
and an idolatrous system of worship
enforced.\sidenote{\nameref{ch:daniel8}:13, \nameref{ch:daniel9}:27, 
\nameref{ch:daniel11}:31} As a result of this attempt to obliterate the knowledge and
worship of the true God, many Jews would apostatize and enter into a
``covenant'' with their oppressor.\sidenote{\nameref{ch:daniel8}:12-13,
\nameref{ch:daniel9}:27, \nameref{ch:daniel11}:22}

The duration of this time of anguish for God's people is given variously as
\begin{enumerate}
    \item ``a time, two times, and half a time'' = three and a half years,
        \sidenote{\nameref{ch:daniel7}:25, \nameref{ch:daniel12}:7} as
    \item the last half of the seventieth of the seventy ``weeks'' = also three and a
        half years,\sidenote{\nameref{ch:daniel7}:25, \nameref{ch:daniel9}:27, 
        \nameref{ch:daniel12}:7} and as 
    \item  the time during which 2300 evening and morning
sacrifices would normally have been offered = 1150 literal days = three
        years, two months, and 10 days\sidenote{\nameref{ch:daniel8}:14} within 
        the three and a half years of ``anguish.''\sidenote{\nameref{ch:daniel9}:27,
        \nameref{ch:daniel12}:1, 7}
\end{enumerate}

At the close of this time of anguish the Ancient of days would sit in
judgment and ``the decreed end'' would be ``poured out upon the desolator,'' who
would thus ``come to his end with no one to help him'' and be ``broken'' but
``not by human hands.``\sidenote{\nameref{ch:daniel7}:22, 66, \nameref{ch:daniel8}:25,
\nameref{\ch:daniel9}:27, \nameref{ch:daniel11}:45, \nameref{ch:daniel12}:11}
Simultaneously, the sanctuary would ``be restored to
its rightful state,`` the Ancient of Days would vindicate His faithful people
and award them an ``everlasting kingdom,`` Michael would arise to deliver
them, the righteous dead would be raised to life eternal, the ``wise,''
including Daniel, would enter upon their eternal reward and shine like the
brightness of the firmament for ever and ever.\sidenote{\nameref{ch:daniel7}:22, 27,
\nameref{ch:daniel8}:14, \nameref{ch:daniel12}:1-3, 13-14}

The prophecies of Daniel locate this time of anguish 
\begin{enumerate}
    \item during the ``time,
two times, and half a time'' of Daniel 7:25, 
    \item at or near ``the end'' of the
``rule'' of the four horn Greek era of chapter 8:8, 21-23, 
    \item during the last
half of the seventieth of the seventy weeks of chapter 9:24-27, and 
    \item during the reign of the last king of the north of chapter 11:20-45. 
\end{enumerate}

Obviously Daniel's perspective of salvation history was vastly different
from ours--by more than two thousand years! But by the sure word of his
angel mentor that was the perspective from which he and the angel Gabriel
then viewed the future. It is the identical format set forth in the Old
Testament.\sidenote{See note 35} To ignore or deny it is a major violation of the sola Scriptura
principle, and to say that neither Daniel nor Gabriel knew what they were
talking about! It is an important part of in-depth study of the Bible to 
read it from its own historical and salvation history perspectives, in order
to understand and appreciate its message for us in our time!

Daniel's perspective of salvation history thus explicitly invalidates the
historicist concept of predictive prophecy. Furthermore, his perspective was
identical with that of the Old Testament as a whole.\sidenote{See note 35}

\subsection{Four KJV Translation Errors That Led Pioneer Adventists Astray}

Four major translation errors in the KJV of Daniel 8:14 and 9:25-26, of
which William Miller and pioneer Adventists were obviously unaware, led
them, unwittingly, astray.\sidenote{enumerated below}

The KJV of Daniel 8:14 reads: ``Unto two thousand and three hundred days;
then shall the sanctuary be cleansed.'' Here and in chapter 9 the KJV
inaccurately reflects the Hebrew text of Daniel at four specific points. In
the original Hebrew text and in the NRSV it reads: ``For two thousand and
three hundred evenings and mornings; then the sanctuary shall be restored
to its rightful state''.

The Hebrew word for ``days,'' \textit{yamim}, is not in the Hebrew text of 8:14, which
reads simply \textit{erev boquer}, ``evening morning.'' ``Days'' is interpretation, not
eranslation. When Daniel meant ``days'' he consistently wrote ``days'',
\textit{yamim}.\sidenote{Daniel 1:12, \nameref{ch:daniel8}:26-27, \nameref{ch:daniel10}:13-14,
\nameref{ch:daniel11}:20, \nameref{ch:daniel12}:11-12} Wherever the words \textit{erev} and \textit{boquer} occur in a sanctuary context
(as in 8:14), without exception they always refer to the evening and morning
sacrificial worship services or to some other aspect of the sanctuary and
its ritual services. These sacrifices were offered \textit{tamid}, ``regularly,'' late
every afternoon before sunset and early every morning after sunrise. See,
for example, Exodus 29:38-42 and Numbers 28:3-6. \textit{Erev} sometimes precedes
\textit{boquer} in view of the fact that Hebrew custom began each day at sunset, with
\textit{erev} referring specifically to the waning light of day associated with
sunset and \textit{boquer} the rising light of day associated with sunrise, not to
the dark and light portions of a 24-hour day.

The traditional interpretation considers \textit{erev boquer}, ``evening morning,'' a
composite term meaning a 24-hour day. But according to verse 26 \textit{haerev we
haboquer}, ``the evening and the morning,'' are discrete entities, as the
repeated definite article requires. The question of verse 13, and thus the
answer of verse 14 both focus on the sanctuary and the time during which its
continual (\textit{tamid}) burnt offering was banned. Accordingly, \textit{erev boquer} in
verse 14 is to be understood in a cultic sanctuary context specifically with
reference to the \textit{tamid} (continual) burnt offering.

Note also that the question of verse 13, to which verse 14 is the inspired
answer, asks for how long the \textit{tamid}, the ``regular burnt offering'' already
mentioned in verse 11, would be ``trampled.'' In place of \textit{tamid} in verse 13, 
however, verse 14 substitutes the expression \textit{erev boquer}, thereby calling
attention to the fact that the two are synonymous terms for the same thing,
the evening and morning sacrificial worship services. Indeed, both terms
occur together in the passages noted above with respect to the two daily
worship services. (In 8:11 and 14 the NRSV--correctly--adds
``burnt offering'' to the term ``regular,'' \textit{tamid}, in recognition of the fact
that \textit{tamid} refers to the daily, or regular, burnt offerings.)

The word tamid, ``continual(ly),'' ``regular(ly),'' occurs 104 times in the Old
Testament, 51 times in connection with the sanctuary ritual, 53 times
otherwise. More than half of the 51 sanctuary-related occurrences are in
connection with the daily burnt offering (32 of the 51 times); and 19 times
of the bread of the presence, the lamp, the cereal offering, and other
aspects of the sanctuary and its ritual. 

The Hebrew word \textit{nitsdaq} never means ``cleansed,'' as the KJV translates it.
\textit{Nitsdaq} is the passive form of the verb \textit{tsadaq}, ``to be right,'' and means ``to
be set right,'' or as the NRSV renders it, ``to be restored to its rightful
state.'' Had Daniel meant ``cleansed'' he would have used the word \textit{taher}, which
does mean ``cleansed'' and always refers to ritual cleansing in contrast to
\textit{tsadaq}, which always connotes moral rightness.\sidenote{As in \nameref{ch:lev16}} 

Daniel 8:14 is concerned with the meaning of the sacrificial worship
service, not with whether it was performed correctly. It affirmed Israel's
continued loyalty to God and commitment to its covenant relationship with
Him, at the beginning and again at the close of each day. The KJV based its
rendering of \textit{nitsdaq} as ``cleansed'' on the Latin Vulgate, which reads
\textit{mundabitur}, and the Greek Septuagint, which reads \textit{katharisthesetai}, both of
which denote ritual cleansing, probably reflecting the ritual cleansing of
the temple after its desecration by Antiochus IV Epiphanes in 167 B.C., as
recorded in 1 Maccabees 4:36-54.\sidenote{A comparison of the career of Antiochus IV Epiphanes as set forth in
1 Maccabees 1 to 4 with the little horn of Daniel results in 24 points of
undeniable identity. This led ancient Jewish scholars to identify him as the
fulfillment of the Daniel's predictions. However, Christ's statements in
Mark 1:15, Matthew 24 (etc.), and some forty times by New Testament writers
locate the fulfillment of Daniel's end-time prophecies at the close of New
Testament times. See references cited in Notes 130 and 131.}

The KJV's ``the Messiah the Prince'' in Daniel 9:25 and ``Messiah'' in verse 26,
respectively, constitute interpretation of the Hebrew text, not translation
of it. The Hebrew text reads ``an anointed, a prince'' or ``an anointed prince''
in 9:25 and ``an anointed'' in verse 26. In so doing, the KJV commits a double
error by: 
\begin{enumerate}
	\item rendering the Hebrew indefinite as definite, and 
	\item arbitrarily identifying the anointed prince as Jesus Christ.
\end{enumerate}
This double
error automatically led pioneer Adventists to another, even grosser, error
in verse 27, considered below.

To be sure, the English word ``messiah'' accurately transliterates the Greek
\textit{messias}, which in turn transliterates the Hebrew \textit{mashshiach}, and the English
word ``Christ'' accurately translates the Greek \textit{messias}. But the KJV
translators had no legitimate reason for rendering the Hebrew indefinite as 
definite and identifying the anointed prince of Daniel 9:25 and 26 as Jesus
Christ.

The KJV rendering ``seven weeks, and three score and two weeks'' in 9:25,
implying a total of sixty-nine ``weeks'' between ``the going forth of the
commandment to restore and to build Jerusalem'' and the coming of its
``Messiah the Prince,'' grossly misrepresents the Hebrew syntax of verse 25.

Hebrew syntax requires that the seven-week period be the time between the
``going forth of the commandment to restore and to build Jerusalem'' and the
``anointed prince'' referred to, and that the ``threescore and two weeks'' refer
to the duration of the ``troublous times'' during which the ``street'' and the
``wall'' remain built prior to the evil ``prince that shall come'' of the
following verse. The NRSV renders the Hebrew syntax of verse 25 correctly:
``\ldots there shall be seven weeks; and for sixty-two weeks it [Jerusalem]
shall be built again \ldots'' Verse 26 confirms the fact that the seven weeks
and the sixty-two weeks are two discrete periods of time, not one composite
time period. Hebrew usage throughout the Old Testament confirms this
conclusion.

Those who formulated the traditional Adventist interpretation of Daniel 8:14
were led astray by these four KJV errors. Had they been working directly
from the Hebrew text of Daniel, or an accurate English translation, they
would never have contrived the traditional Adventist interpretation.

Their second error was adoption of the day-for-a-year interpretation of
Bible prophecy. That pseudo principle, inherent in the historicist
interpretation of Bible prophecy, was invented in the ninth century by the
Jewish scholar Nahawendi, as a device by which to make Daniel's prophecies
relevant to his day. Catholic scholars subsequently adopted and used it
until certain other Catholic scholars, and later Protestants, based
their identification of the papacy as the antichrist of Bible prophecy on
it. Thereupon Roman Catholics abandoned the day-for-a-year principle,
whereas Protestants retained it as proof that Rome was ``Babylon.'' Suffice it
to note, here, that there is no Bible basis whatever for this so-called
principle.\sidenote{The prophetic day-for-a-literal-year concept was originally
formulated by the Karaite Jewish scholar Nahawendi in the ninth century in
an endeavor to identify events of his time as the fulfillment of Daniel's
prophecies. The idea that this ``principle'' was operative with respect to the
seventy ``weeks'' of years of Daniel 9 ignores the fact that it was, as a
matter of fact, an application of the ancient Jewish jubilee-year system of
dating, not the purported day-for-a-year ``principle.'' The ancient Jewish
Book of Jubilees uses this system of dating scores of times for dating
events in Jewish history. See Chapter 15, ``Jewish Interpretation of Daniel'',
in my Eschatology of Daniel for a number of relevant examples from the Book
of Jubilees. See also \bibentry{42b}, pp. 52-55, 208; 
\bibentry{102c}, p. 713; \bibentry{102d}, p. 196}

\subsection{The Immediate Context of Daniel 8:14}

The vision of chapter 8:1-12, the question of verse 13, and the explanation
of verses 15 to 27 constitute the immediate context of verse 14. As a matter
of fact chapter 8 itself identifies all four essential elements of verse 14:
\begin{enumerate}
	\item its sanctuary, 
	\item why it needed cleansing or being ``restored to its
rightful state,'' 
	\item how long it had needed cleansing or restoration, and
	\item when that cleansing or restoration would occur.
\end{enumerate}

According to verses 9-12, their cryptic little horn invades the ``beautiful
land'' and overthrows the sanctuary located there--obviously the sanctuary,
or temple, in Jerusalem. Verse 14 itself specifies that the period of time
during which the sanctuary would remain overthrown and its regular burnt
offering suspended as the time during which 2300 ``regular burnt offerings''
would normally have been offered. With two such offerings each day, that
would be1150 literal twenty-four-hour days, or three years, two months, and
ten days. When would this occur? Verses 21 to 25 specify that all of this,
including the cleansing or restoration of the sanctuary to its rightful
state, would take place soon after the close of the four-horn (Hellenistic)
Greek era of the prophecy.

Verse 13, the question to which verse 14 is the answer, identifies the
``evenings and mornings'' as an equivalent term for its ``regular burnt
offering.''104 The nature of the sanctuary's cleansing or restoration is
explained in the proximate context of the rest of the Book of Daniel, which
also identifies other events that accompany or follow its cleansing or
restoration.

Verses 11 and 12 of chapter 8 attribute the trampling of the sanctuary
mentioned in verses 11-13 to the cryptic little horn of verse 8, which
verses 21 to 23 identify as ``a king of bold countenance'' at ``the end'' of the
four horn (Greek) era of the vision. Accordingly, context explicitly
identifies the restoration of the sanctuary to its rightful state in verse
14 as removal of the damage caused by the little horn. The sanctuary's
overthrown, trampled state included, particularly, the taking away of its
``regular burnt offering'' and substitution of the ``transgression that makes
desolate''105 in its place.

The answer of verse 14 substitutes the expression ``evenings and mornings''
for verse 13's question about ``the regular burnt offering,'' thereby
identifying them as equivalent terms for the same thing. With two such
sacrifices each day, the time during which 2,300 evening and morning
sacrifices would normally have been offered would be a period of 1,150
literal days, or nearly three and a half literal years. Verse 26 identifies
the time in history when this would happen as the ``appointed time of the end
\ldots many days from now,'' ``at the end'' of the ``rule'' of the four Greek
(Hellenistic) horns of the male goat.106

The immediate context of verse 14--chapter 8 itself--thus identifies all
of the essential elements of the verse, but leaves the restoration of
the sanctuary ``to its rightful state'' unexplained because Daniel fell
ill.107 As will be seen, events associated with that restoration are
revealed elsewhere in Daniel. The traditional Adventist interpretation of
Daniel 8:14 thus removes it completely from the immediate context in which
Gabriel and Daniel placed it, in obvious violation of the sola Scriptura 
principle. The proximate context--Daniel 7, 9, and 10-12--clarifies
matters still further.

\subsection{Daniel 9 as Proximate, Continuing Context for 8:14}

The traditional Adventist interpretation of Daniel 8:14 recognizes a
relationship between chapters 8 and 9, but at three vital points
misconstrues its contextual contribution to an accurate understanding of
8:14. This valid relationship is evident from
\begin{enumerate}
	\item the fact that
Gabriel had not been able to complete his commission to explain the vision
of chapter 8,108 
	\item that when he reappears in 9:21-25 he summons Daniel to
``understand'' that vision, and 
	\item that his message in 9:24-27 provides the
very information needed to complement his aborted explanation of 8:19-27.
\end{enumerate}

The traditional interpretation assumes that the 70 ``weeks'' of years of 9:24
constitute the first 490 of its 2300 \textit{erev boquer} construed as that many
literal years during which the sanctuary is said to be desolate. But
according to 9:24-26 the sanctuary is restored and in full operation during
the first 69 of the 70 ``weeks''! How can the same sanctuary be restored and
in full operation109 during the very time 8:13-14 has it ``desolate''? This
insoluble paradox, inherent in and indispensable to the traditional
interpretation, constitutes it an oxymoron!

The second contextual anomaly implicit in and essential to the traditional
interpretation is its identification of the \textit{davar}, ``word'' (KJV
``commandment''), that went out to restore and build Jerusalem,110 as the
decree of Artaxerxes Longimanus in 457 B.C. But that decree111 says nothing
about rebuilding either Jerusalem or the temple, which had already been
rebuilt and in operation for 59 years!112

Immediately prior to Gabriel's reappearance and message recorded in 9:20-27
Daniel had been pleading in prayer for God to restore His now desolate
sanctuary in Jerusalem.113 At this point in Daniel's prayer Gabriel
interrupts to announce that a \textit{davar}, ``word''114 (or ``command,'' KJV) had
already gone forth, obviously in heaven, in response to his prayer, and that
he (Gabriel) had now come to ``declare it'' to Daniel. He forthwith repeats
that ``word''115 and explains it.116 Contextually, the ``word'' that ``went out
[\textit{motsa}] to restore and rebuild Jerusalem''117 is the very ``word'' that ``went
out'' (\textit{yatsa}) in response to Daniel's prayer,118 and is quoted verbatim in
verse 24! Gabriel assures Daniel that God Himself, not some earthly monarch,
had already answered his fervent prayer! Obviously that ``word''119 is one
that only God Himself could possibly have issued, not some earthly monarch!

With considerable support even among presumably reputable Bible scholars,
the traditional Adventist interpretation identifies the ``he'' of 9:27 who
``make[s] a strong covenant with many'' renegade Jews for the seventieth of 
the seventy weeks,120 and for half of the week makes ``sacrifice and
offering cease,'' as the ``Messiah the Prince'' (KJV) of verses 25 and 26,
meaning Christ. But the immediate antecedent of the pronoun ``he'' in verse 27
is the evil ``prince that shall come'' of verse 26, not the anointed prince of
verse 25! Only reliance on the faulty KJV identification of the anointed
prince of verse 25 as Christ, and identifying Him as the ``he'' of verse 27,
is the traditional interpretation able to reckon backwards to identify the
decree of Artaxerxes Longimanus in 457 B.C. as marking the beginning of the
seventy ``weeks'' of years (and thus also of its 2300 years). Furthermore, the
Hebrew \textit{ein lo} of verse 26 (KJV ``but not for himself,'' NRSV ``shall have
nothing'') actually means that the cut off prince would have no successor.
Thus to have either him or a successor reappear as the ``he'' of verse 27
makes verse 27 contradict verse 26! Another oxymoron!

Identifying the ``he'' of verse 27 as the evil ``prince who is to come'' of
verse 26, however, makes verse 27 an exact parallel to the career of the
little horn in chapter 8, who likewise ``makes sacrifice and offering cease''
and in their place sets up ``an abomination that desolates.''121 Remember, as
pointed out above, that the angel Gabriel specifically presented 9:25-27 as
a continuing explanation of the prophecy of chapter 8. To complete the
parallel, he now122 tells Daniel that ``the decreed end is poured out upon
the desolator,'' as he had formerly told him (in chapter 8) that ``the king of
bold countenance'' would ``be broken, and not by human hands.''123

This contextual understanding of 9:27 automatically and conclusively locates
the 2300 ``evenings and mornings'' of 8:14, understood as the number of
sacrifices that would normally be offered, two each day, during the course
of 1150 days, within the 1260 days, or three and a half years of the last
half of the seventieth ``week'' of years of chapter 9--the ``appointed time of
the end'' in the ``latter part'' of the four-horn era124 when the little horn
of verses 9-13, 23-27 appears on the prophetic stage in what was, in
Daniel's time, ``the distant future.''125
