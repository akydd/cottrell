\chapter{``Rightly Explaining'' Daniel 8:14}
\label{ch:explain}

\newthought{The first imperative} for comprehending the prophecies of Daniel in the sense
Inspiration intended is an objective frame of mind divested of every
personal, subjective, modern presupposition with respect to their import.

The second imperative is to identify the circumstances set forth in Daniel 1
to 6 and 9:1-23, which provide the historical background within which
Inspiration set its five prophetic passages and from which it intended
Daniel and his intended readers to understand them. Accordingly, in order to
understand those passages as Inspiration intended them to be understood we
must do so with that historical perspective in our minds, and from the same
perspective of salvation history as Daniel and his intended readers did. Any
interpretation that ignores or controverts that historical perspective and /
or the salvation history perspective of their time is automatically suspect

The first six chapters of the Book of Daniel recount the exile of Daniel and
his compatriots to Babylon ``in the third year of the reign of Jehoiakim of
Judah,'' which is dated to 606/5 B.C., and their experiences during the
seventy years of exile foretold by Jeremiah in chapter 29:1-14. According to
Daniel 9:1, in ``the first year of Darius'' (which is dated to 537/6 B.C. by
Jewish inclusive reckoning), Daniel had been in exile for exactly seventy
years. But as yet there was no visible evidence that release from exile was
imminent. Accordingly, Daniel prayed the importunate prayer for release from
exile and for restoration recorded in chapter 9:4-19. 
and imposes an alien, uninspired interpretation on those prophecies. 

While Daniel was still praying the angel Gabriel 
reappeared\sidenote{\nameref{ch:daniel9}:23, cf. \nameref{ch:daniel8}:16} and said, ``I
have now come out to give you wisdom and understanding. At the beginning of
your supplications a word went out [obviously in heaven], and I have come to
declare it, for you are greatly beloved. So consider the word and understand
the vision.'' Gabriel thereupon repeats that ``word'' verbatim (verse 24), as
he had promised, and proceeds to explain it in verses 25 to 27. 

It is of crucial importance to note that Gabriel explicitly identifies the
``word'' that ``went out to restore and build Jerusalem'' at the commencement of
the seventy weeks of years as ``the word'' that ``went out''--in heaven--while
Daniel was praying.\sidenote{\nameref{ch:daniel9}:21-23} That 
``word''\sidenote{\nameref{ch:daniel9}:24} was obviously one that only God Himself 
(and not an earthly monarch) could possibly have issued! On the authority of
no less a person than the angel Gabriel, the ``seventy weeks'' of years thus
began in 537 B.C., not eighty years later in 457 B.C.!

Gabriel's explanation of that ``word'' in verses 25--27 very briefly sketched
the future of God's covenant people during the seventy weeks of years, and
its climax in the ruthless oppression of ``the prince who is to come'' during
the seventieth of the seventy ``weeks,'' which he had already foretold in
chapter 8:9--13 and explained in verses 19 to 
25.\sidenote{cf. \nameref{ch:daniel7}:24-25}

As already noted, Daniel 9:23-25 begins the seventy weeks of years at the
time the ``word'' was issued in heaven, in 537 B.C. In the same way,
contextual identification of the ``he'' of verse 27 identifies events of
history that mark their close in the seventieth of the seventy ``weeks.'' It
is universally accepted that the immediate antecedent of a personal pronoun
identifies the person to whom it refers unless the context unambiguously
specifies otherwise. Accordingly, verse 26 identifies the immediate
antecedent of the pronoun ``he'' in verse 27, who ``make[s] a strong covenant
with many'' for the seventieth of the seventy ``weeks'' and ``make[s] sacrifice
and offering cease'' during the last half of the ``week,'' as the evil ``prince
who is to come''--not the ``anointed prince'' of verses 25--26! 

Chapter 11:23 confirms the fact that his alias, the last king of the north,
does, indeed make such a covenant with people in ``alliance'' with him. Also,
his fate set forth in verse 27, ``the decreed end is poured out on the
desolator,'' is equivalent to the horn-king of chapter 8:25 being ``broken,
and not by human hands,'' and to the last king of the north in chapter 11 who
``come[s] to his end, with no one to help him.''\sidenote{\nameref{ch:daniel11}:45} 

Chapter 9:24--27 thus provides an exact but much more complete explanation of
chapter 8:13-14's question and answer about events between Daniel's time and
``the appointed time of the end'' ``many days from now'' when ``the vision of the
evenings and the mornings'' was to meet its 
fulfillment.\sidenote{\nameref{ch:daniel8}:17, 26} Isn't that exactly
what Gabriel said the audition of 9:24-27 was supposed to 
do?\sidenote{\nameref{ch:daniel9}:22-25} 

Such is Daniel's perspective of salvation history. In order to understand
chapters 8 and 9 as heaven intended them to be understood, we must imagine
ourselves in Daniel's historical circumstances and view them from his
perspective of salvation history in order to form an accurate understanding
of what was revealed to him.

\section{Daniel's Perspective of Salvation History}

Daniel's perspective of salvation history was a composite of the visions of
chapters 2 and 7, each with its explanation, and chapter 8 with
its three-fold explanation in chapters 8, 9, and 11-12. It consisted of a 
series of universal kingdoms\sidenote{\nameref{ch:daniel2}:37-40, \nameref{ch:daniel7}:3-7,
\nameref{ch:daniel8}:3-8, \nameref{ch:daniel11}:2-3} followed by a period of disintegration and
fragmentation,\sidenote{\nameref{ch:daniel2}:41-43, \nameref{ch:daniel7}:7-8, 17, 23,
\nameref{ch:daniel8}:8-9, \nameref{ch:daniel8}:8-9, \nameref{ch:daniel11}:4-5, 25-29,
40-43} which Gabriel told Daniel would be a ``troubled time''
(9:25)\sidenote{\nameref{ch:daniel9}:25}

At the ``appointed time of the end \ldots many days from now''--after sixty-nine
of the ``seventy weeks of years''\sidenote{\nameref{ch:daniel2}:44, \nameref{ch:daniel7}:28,
\nameref{ch:daniel8}:17, 19, 26, \nameref{ch:daniel9}:24, 27, \nameref{ch:daniel11}:35, 40}
--there would be an unprecedented ``time of
anguish'' for God's people in which they would be ``trampled,'' their power
shattered,\sidenote{\nameref{ch:daniel7}:21, 25, \nameref{ch:daniel8}:10, 13, 24-25,
\nameref{ch:daniel9}:26, \nameref{ch:daniel12}:1, 2, 7} their land and city 
devastated,\sidenote{\nameref{ch:daniel8}:9, \nameref{ch:daniel9}:36,
\nameref{ch:daniel11}:22, 24, 41} their loyalty and
faithfulness to God tested,\sidenote{\nameref{ch:daniel8}:11, 25, \nameref{ch:daniel11}:36} 
their covenant with Him and its prescribed
system of worship abolished,\sidenote{\nameref{ch:daniel7}:25, \nameref{ch:daniel8}:11-12,
\nameref{ch:daniel9}:26-27, \nameref{ch:daniel11}:31, \nameref{ch:daniel12}:11}
and an idolatrous system of worship
enforced.\sidenote{\nameref{ch:daniel8}:13, \nameref{ch:daniel9}:27, 
\nameref{ch:daniel11}:31} As a result of this attempt to obliterate the knowledge and
worship of the true God, many Jews would apostatize and enter into a
``covenant'' with their oppressor.\sidenote{\nameref{ch:daniel8}:12-13,
\nameref{ch:daniel9}:27, \nameref{ch:daniel11}:22}

The duration of this time of anguish for God's people is given variously as
\begin{enumerate}
    \item ``a time, two times, and half a time'' = three and a half years,
        \sidenote{\nameref{ch:daniel7}:25, \nameref{ch:daniel12}:7} as
    \item the last half of the seventieth of the seventy ``weeks'' = also three and a
        half years,\sidenote{\nameref{ch:daniel7}:25, \nameref{ch:daniel9}:27, 
        \nameref{ch:daniel12}:7} and as 
    \item  the time during which 2300 evening and morning
sacrifices would normally have been offered = 1150 literal days = three
        years, two months, and 10 days\sidenote{\nameref{ch:daniel8}:14} within 
        the three and a half years of ``anguish.''\sidenote{\nameref{ch:daniel9}:27,
        \nameref{ch:daniel12}:1, 7}
\end{enumerate}

At the close of this time of anguish the Ancient of days would sit in
judgment and ``the decreed end'' would be ``poured out upon the desolator,'' who
would thus ``come to his end with no one to help him'' and be ``broken'' but
``not by human hands.``\sidenote{\nameref{ch:daniel7}:22, 66, \nameref{ch:daniel8}:25,
\nameref{\ch:daniel9}:27, \nameref{ch:daniel11}:45, \nameref{ch:daniel12}:11}
Simultaneously, the sanctuary would ``be restored to
its rightful state,`` the Ancient of Days would vindicate His faithful people
and award them an ``everlasting kingdom,`` Michael would arise to deliver
them, the righteous dead would be raised to life eternal, the ``wise,''
including Daniel, would enter upon their eternal reward and shine like the
brightness of the firmament for ever and ever.\sidenote{\nameref{ch:daniel7}:22, 27,
\nameref{ch:daniel8}:14, \nameref{ch:daniel12}:1-3, 13-14}

The prophecies of Daniel locate this time of anguish 
\begin{enumerate}
    \item during the ``time,
two times, and half a time'' of Daniel 7:25, 
    \item at or near ``the end'' of the
``rule'' of the four horn Greek era of chapter 8:8, 21-23, 
    \item during the last
half of the seventieth of the seventy weeks of chapter 9:24-27, and 
    \item during the reign of the last king of the north of chapter 11:20-45. 
\end{enumerate}

Obviously Daniel's perspective of salvation history was vastly different
from ours--by more than two thousand years! But by the sure word of his
angel mentor that was the perspective from which he and the angel Gabriel
then viewed the future. It is the identical format set forth in the Old
Testament.35 To ignore or deny it is a major violation of the sola Scriptura
principle, and to say that neither Daniel nor Gabriel knew what they were
talking about! It is an important part of in-depth study of the Bible to 
read it from its own historical and salvation history perspectives, in order
to understand and appreciate its message for us in our time!

Daniel's perspective of salvation history thus explicitly invalidates the
historicist concept of predictive prophecy. Furthermore, his perspective was
identical with that of the Old Testament as a whole.98

\subsection{Four KJV Translation Errors That Led Pioneer Adventists Astray}

Four major translation errors in the KJV of Daniel 8:14 and 9:25-26, of
which William Miller and pioneer Adventists were obviously unaware, led
them, unwittingly, astray.99

The KJV of Daniel 8:14 reads: ``Unto two thousand and three hundred days;
then shall the sanctuary be cleansed.'' Here and in chapter 9 the KJV
inaccurately reflects the Hebrew text of Daniel at four specific points. In
the original Hebrew text and in the NRSV it reads: ``For two thousand and
three hundred evenings and mornings; then the sanctuary shall be restored
to its rightful state''.

The Hebrew word for "days," yamim, is not in the Hebrew text of 8:14, which
reads simply erev boquer, "evening morning." "Days" is interpretation, not
eranslation. When Daniel meant "days" he consistently wrote "days,"
yamim.100 Wherever the words erev and boquer occur in a sanctuary context
(as in 8:14), without exception they always refer to the evening and morning
sacrificial worship services or to some other aspect of the sanctuary and
its ritual services. These sacrifices were offered tamid, "regularly," late
every afternoon before sunset and early every morning after sunrise. See,
for example, Exodus 29:38-42 and Numbers 28:3-6. Erev sometimes precedes
boquer in view of the fact that Hebrew custom began each day at sunset, with
erev referring specifically to the waning light of day associated with
sunset and boquer the rising light of day associated with sunrise, not to
the dark and light portions of a 24-hour day.

The traditional interpretation considers erev boquer, "evening morning," a
composite term meaning a 24-hour day. But according to verse 26 haerev we
haboquer, "the evening and the morning," are discrete entities, as the
repeated definite article requires. The question of verse 13, and thus the
answer of verse 14 both focus on the sanctuary and the time during which its
continual (tamid) burnt offering was banned. Accordingly, erev boquer in
verse 14 is to be understood in a cultic sanctuary context specifically with
reference to the tamid (continual) burnt offering.

Note also that the question of verse 13, to which verse 14 is the inspired
answer, asks for how long the tamid, the ``regular burnt offering'' already
mentioned in verse 11, would be ``trampled.'' In place of tamid in verse 13, 
however, verse 14 substitutes the expression erev boquer, thereby calling
attention to the fact that the two are synonymous terms for the same thing,
the evening and morning sacrificial worship services. Indeed, both terms
occur together in the passages noted above with respect to the two daily
worship services. (In 8:11 and 14 the NRSV---correctly---adds
``burnt offering" to the term ``regular,'' tamid, in recognition of the fact
that tamid refers to the daily, or regular, burnt offerings.)

The word tamid, ``continual(ly),'' ``regular(ly),'' occurs 104 times in the Old
Testament, 51 times in connection with the sanctuary ritual, 53 times
otherwise. More than half of the 51 sanctuary-related occurrences are in
connection with the daily burnt offering (32 of the 51 times); and 19 times
of the bread of the presence, the lamp, the cereal offering, and other
aspects of the sanctuary and its ritual. 

The Hebrew word nitsdaq never means "cleansed," as the KJV translates it.
Nitsdaq is the passive form of the verb tsadaq, "to be right," and means "to
be set right," or as the NRSV renders it, "to be restored to its rightful
state." Had Daniel meant "cleansed" he would have used the word taher, which
does mean "cleansed" and always refers to ritual cleansing in contrast to
tsadaq, which always connotes moral rightness.101 

Daniel 8:14 is concerned with the meaning of the sacrificial worship
service, not with whether it was performed correctly. It affirmed Israel's
continued loyalty to God and commitment to its covenant relationship with
Him, at the beginning and again at the close of each day. The KJV based its
rendering of nitsdaq as "cleansed" on the Latin Vulgate, which reads
mundabitur, and the Greek Septuagint, which reads katharisthesetai, both of
which denote ritual cleansing, probably reflecting the ritual cleansing of
the temple after its desecration by Antiochus IV Epiphanes in 167 B.C., as
recorded in 1 Maccabees 4:36-54.102\\


