\chapter{The Sanctuary Doctrine and Sola Scriptura}
\label{ch:sola}

The traditional Adventist sanctuary doctrine is based on the historicist
principle, or method, of prophetic interpretation. Consequently, those who
follow that method automatically find the doctrine flawless. On the other
hand, those who follow the historical principle, or method, find it
bristling with flaws. As a result, differences of opinion with respect to 
the sanctuary doctrine can be resolved only by objectively testing the
presuppositions and methodology on which it is based, by the sola Scriptura
principle. The two methods are as mutually exclusive and irreconcilable as
day and night, and a choice between them is decisive for the study of Bible
prophecy.

Historicism is based on the untested pre-concept that the modern reader's
perspective of salvation history is inherent in Bible prophecy and therefore
in full harmony with the sola Scriptura principle. According to the
historicist principle the modern reader of the Bible is to understand its
statements with respect to the end time of human history and associated
events, in terms of our modern perspective of salvation history, with an
uninterrupted, continuous fulfillment of Bible prophecy throughout the two
thousand years since Bible times. The sanctuary doctrine and its advocates
have always taken this principle for granted and never tested its presumed
validity objectively, that is, by the Bible itself. This was true at Glacier
View in August 1980. It is equally true of the subsequent GC-appointed
Daniel and Revelation Committee and its seven-volume official report, which
presupposes the inherent validity of historicism but never attempts to test
or defend it objectively by the sola Scriptura principle.

On the other hand, the historical principle begins with objective attention
to prophetic statements of the Bible in terms of their import as determined
by the historical circumstances and salvation history perspective within
which they were given and to which they were intended to apply. This
principle is not adopted as a subjective pre-concept, but on the objective
basis of plain sola Scriptura evidence, as illustrated in chapters 
\nameref{ch:rightly} and \nameref{ch:explain}
above with respect to Daniel's own explicit historical and salvation history
perspective. Both are inherent in the Book of Daniel and obvious when read
objectively.

Chapter \nameref{ch:explain} above examines the historical sections of the Book of Daniel and
Daniel's own perspective of salvation history with the objective of
determining the historical circumstances and salvation history perspective
as a basis for understanding the import of its prophetic sections. Daniel's
salvation history perspective is identical with that of the Old Testament as
a whole, as my article ``The Role of Israel in Old Testament 
Prophecy''\noteXXXV~in
volume 4 of the SDA Bible Commentary demonstrates. Chapter 4, ``The Old Testament
Perspective of Salvation History'' of my 725-page \bibentry{20a}
provides replete Bible evidence for the
conclusion that it anticipates the climax of human history at the close of
Old Testament times, or soon thereafter.

Jesus and the New Testament writers unanimously reiterate this Old Testament
perspective of salvation history and anticipate His promised return at the
climax of New Testament times. In 36 pages chapter 12 of \bibentry{20a},
``The New Testament Perspective of Salvation History,'' covers this
aspect of the subject in considerable detail.

In summary, at the beginning of His public ministry Jesus announced as the
theme of His mission: ``The time is fulfilled, and the kingdom of God has
come near, repent and believe in the good news.'' What was fulfilled? The
time prophecies of Daniel, alone in the Old Testament, identify the ``time''
to which Jesus here refers. Thus, on no less than the authority of Jesus
Himself, fulfillment of the ``time'' specified by Daniel was near when Jesus
appeared in fulfillment of Old Testament anticipation of His coming. During
the course of His sermon in the synagogue at Nazareth He declared concerning
the Messianic prophecy of Isaiah 61:1--3: ``Today this scripture has been
fulfilled in your hearing.''

During the course of Jesus' response to the disciples' inquiry concerning
the destruction of the Temple, to which He had just referred, the ``sign'' of
His promised return and ``the end of the age'' was, ``When you see the
desolating sacrilege standing in the holy place spoken of by the prophet
Daniel \ldots know that he is near, at the very gates. Truly I tell you, this
generation will not pass away until all these things, [specifically
including His coming in the clouds of heaven to gather His elect] have taken
place.''\sidenote{\nameref{ch:matt24}:1--3, 30--34}

That Jesus specifically intended His remarks concerning the prophecy of
Daniel being fulfilled in His disciples' own generation is evident from 
\begin{enumerate}
\item
His use of the pronouns ``you'' and is His disciples' generation is
evident from His repeated ``your'' twelve times throughout His discourse, and 
\item
their repeated use of such expressions as ``the end of the times,'' ``the
coming of the Lord is at hand,'' ``it is the last hour,'' ``these last days,''
``the time is near,'' ``He is coming soon,'' ``the time has grown very short,''
``the end of the ages has come,'' ``these last days,'' and ``yet a little while,''
nearly forty times when referring to Jesus' anticipated return.\sidenote{See \nameref{ch:shp}} John the
revelator specifically says that everything in the Book of Revelation ``must
soon take place,'' and Jesus assures him four times ``I am coming soon,'' and
the last of which, ``surely I am coming soon.''\sidenote{See Rev. 1:1, 3; 3:11;
22:6--7, 12, 20 listed in~\ref{tab:shp-john}}
\end{enumerate}

There is not the slightest suggestion or hint anywhere in either the Old or
the New Testaments that Jesus' return would be postponed more or less
indefinitely beyond Bible times. The Bible evidence is all explicitly to the
contrary. The Bible itself knows nothing whatever about the historicist
interpretation of its prophecies, a concept that is gratuitously imposed
upon them. If Gabriel and Daniel were here today they would inevitably
render the verdict of sola Scriptura against historicism and in favor of a
historical understanding of Bible prophecy, including that of the Book of
Daniel, and insist on the Bible's own historical and salvation history
perspectives! 

The historicist principle by which Adventists have consistently understood
and interpreted Bible prophecy has, ever since the beginning, imposed our
uninspired modern perspective of salvation history on it, and thereby been
in unwitting violation of the sola Scriptura principle. In contrast, the
historical principle honors the Bible's own perspective of salvation
history, within which its prophetic messages were given and to which they
were intended to apply. It thereby consistently honors the sola Scriptura
principle. Let us not soon forget that the historicist interpretation of
Bible prophecy has ever been and continues to be responsible for the loss of
many otherwise dedicated leaders and the defection of uncounted hundreds of
otherwise faithful Seventh-day Adventists. It has, in addition, diverted
considerable time, attention, and substantial resources of the church from
its mission to the world.

Surely it is high time for responsible church leaders to awake to the
situation and do something about it. The obscurantist 1600-page, 5-volume
Daniel and Revelation Committee report on Daniel accepts and consistently
applies the historicist principle to Bible prophecy---officially for the
church. Do we want the twenty-first century to witness the fulfillment of
Christ's promise to return, or do we prefer to repeat our pathetic
historicist past complacently and indefinitely into the future, and thereby
alienate the respect and confidence of biblically literate Adventists and
non-Adventists?
