\chapter{My Personal Encounter With the Sanctuary Doctrine}
\label{ch:personal}

% sidenote 35 is used in multiple places.  Make it easy.
\newcommand{\noteXXXV}{\sidenote{My article \bibentry{35} classifies and summarizes
some five thousand Old Testament passages relating to God's dealings with
Israel under the covenant relationship, including the Old Testament
perspective of salvation history, which culminated in the coming of Messiah
and the establishment of His eternal reign of righteousness at or soon after
the close of Old Testament times. These five thousand passages were
accumulated during the course of teaching the class Old Testament Prophets
for several years at Pacific Union College during the 1940s and 1950s. The
parenthetical sentence on page 38, ``This rule does not apply to those
portions of the book of Daniel that the prophet was bidden to shut up and
seal, or to other passages whose application Inspiration may have limited
exclusively to our own time,'' was added by F. D. Nichol during the editorial
process. He personally agreed with everything in the article and made no
alterations in it, but feared for the adverse reception of the Commentary
except for this \textit{caveat}.}}

\newthought{I first encountered problems} with the traditional interpretation of Daniel
8:14, professionally, in the spring of 1955 during the process of editing
comment on the Book of Daniel for volume 4 of the SDA Bible Commentary. As
a work intended to meet the most exacting scholarly standards, we intended
our comment to reflect the meaning obviously intended by the Bible writers.
As an Adventist commentary it must also reflect, as accurately as possible,
what Adventists believe and teach. But in Daniel 8 and 9 we found it
hopelessly impossible to comply with both of these requirements.\noteXXXV{}

In 1958 the Review and Herald Publishing Association needed new printing
plates for the classic book Bible Readings, and it was decided to revise it
where necessary to agree with the Commentary. Coming again to the Book of
Daniel I determined to try once more to find a way to be absolutely faithful
to both Daniel and the traditional Adventist interpretation of 8:14, but
again found it impossible. I then formulated six questions regarding the
Hebrew text of the passage and its context, which I submitted to every
college teacher versed in Hebrew and every head of the religion department 
in all of our North American colleges---all personal friends of mine.
Without exception they replied that there is no linguistic or contextual
basis for the traditional Adventist interpretation of Daniel 8:14.

When the results of this questionnaire were called to the attention of the
General Conference president, he and the Officers appointed the super-secret
Committee on Problems in the Book of Daniel, of which I was a member.
Meeting intermittently for five years (1961--1966), we considered 48 papers
relative to Daniel 8 and 9, and in the spring of 1966 adjourned sine die,
unable to reach a consensus.\sidenote{My set of the committee papers considered is in the GC
Archives.}

The Commentary experience with Daniel already mentioned led me into an
unhurried, in-depth, spare-time, comprehensive study of Daniel 7 to 12
that continued without interruption for seventeen years (1955--1972), in
quest of a conclusive solution to the sanctuary problem. My objective was to
be fully prepared with definitive, objective, biblical information the next
time the question should arise during the course of my ministry for the
church.

Among other things I memorized, in Hebrew, all relevant portions of Daniel 8
to 12 for instant recall and comparison (60 verses), conducted exhaustive
word studies\sidenote{My study of 150 important words in the Aramaic and Hebrew
portions of Daniel fills 108 typewritten pages.}
of more than 150 relevant Hebrew words Daniel uses,
throughout the Old Testament, studied the Hebrew grammar and syntax in
detail, made a minute analysis of contextual data,\sidenote{My correlation 
of the prophecies of Daniel 7, 8, 9, and
11--12 fills 14 typewritten pages.} compared ancient Greek
and Latin translations of Daniel,\sidenote{For my own convenience, I wrote
out (in parallel columns)
key passages of the prophecies of Daniel in Hebrew, Greek (both the LXX and
Theodotion), the KJV, and RSV.} investigated relevant apocryphal and New
Testament passages,\sidenote{Especially the first four chapters of 1 Maccabees, where I
found twenty-four points of specific identity between Daniel's little horn
and the career of Antiochus IV Epiphanes. I concluded, however, that Christ
assigned the fulfillment of Daniel's prophecies to New Testament times, and
that the New Testament writers nearly forty times anticipate Jesus' promised
return within their generation. Chapters 4 ``The Old Testament Perspective of
Salvation History'' and 12 ``The New Testament Perspective of Salvation
History'' in my book manuscript \bibentry{20a} sets
all of this forth in detail. See Appendix chapter~\nameref{ch:shp}.} traced Jewish and Christian
interpretation of Daniel
from ancient to modern times,\sidenote{Chapter 13 of \bibentry{20a} traces Jewish
interpretation in some detail from ancient to modern times. For this I relied 
primarily on \bibentry{42a}, \bibentry{42aa}, \bibentry{42b}, and \bibentry{42c}.}
and made an exhaustive study of the
formation, development, and subsequent Adventist experience with the
traditional sanctuary doctrine.\sidenote{Chapter 14 of \bibentry{20a} traces
the development of the traditional Adventist 
interpretation of Daniel 8:14 in considerable detail.}
Eventually I incorporated the results of
this investigation into an 1100 page manuscript which I later reduced to 725
pages but decided not release for publication until an appropriate time.
\newpage
The above considerations conclusively demonstrate that our traditional
interpretation of Daniel 8:14, the sanctuary, and the investigative judgment
as set forth in Article 23 of Fundamental Beliefs does not accurately
reflect the teaching of Scripture with respect to the ministry of Christ on
our behalf since His return to heaven.\sidenote{Chapter 17 of \bibentry{20a}
explores its comment on Christ's
ministry in the heavenly sanctuary in considerable detail.} Accordingly, 
it is appropriate
\begin{enumerate}
	\item to note wherein Article 23 is thus defective,\sidenote{see chapter~\nameref{ch:flaws}}
	\item to revise the article so
as to reflect Bible teaching on this aspect of His ministry accurately, and
	\item to suggest a process designed to protect the church from this and
similar traumatic experiences in the future.\sidenote{see chapter~\nameref{ch:remedy}}
\end{enumerate}

Some of the concepts associated with the investigative judgment are, indeed,
biblical, but the Bible itself nowhere associates them with an investigative
judgment, for which there is no sola Scriptura basis whatever.\sidenote{Chapter 17
of \bibentry{20a} explores its comment on Christ's ministry in the heavenly
sanctuary in considerable detail.}

Upon ascending to heaven Jesus assured His disciples ``I am with you always,
to the end of the age'' (Matthew 18:20). The Book of Hebrews is our primary
source of information about His ministry in heaven on their (and our) behalf
since that time, I suggest that the following composite summary of His
ministry as presented in Hebrews provides an appropriate basis for a revised
article 23 of Fundamental Beliefs, should such a statement eventually be
desired. The author of Hebrews presents Christ's ministry in heaven, on our
behalf, by analogy with the role of the high priest in the ancient sanctuary
ritual: 
\begin{quote}
On the cross Jesus offered Himself as a single sacrifice for all time that
atoned for the sins of those who draw near to God through Him.
\sidenote{Heb. 7:27 Unlike the other high priests, he has no need to offer 
sacrifices day after day, first for his own sins, and then for those of the 
people; this he did once for all when he offered himself.
    
Heb. 10:11--12 Day after day every 
priest stands and performs his religious duties; again and again he offers the 
same sacrifices, which can never take away sins. But when this priest had 
offered for all time one sacrifice for sins, he sat down at the right hand of God,}
\newpage~That one
sacrifice qualified Him to serve as our great High Priest in heaven,
perpetually.\sidenote{Heb. 2:17--18 For this reason he had to be made like 
them, fully human in every way, in order that he might become a merciful and 
faithful high priest in service to God, and that he might make atonement for 
the sins of the people. Because he himself suffered when he was tempted, he 
is able to help those who are being tempted.
 
Heb. 4:14--15  Since, then, we have a great high priest who has passed through 
the heavens, Jesus, the Son of God, let us hold fast to our confession. For 
we do not have a high priest who is unable to sympathize with our weaknesses, 
but we have one who in every respect has been tested as we are, yet without sin.

Heb. 6:19--20 We have this hope
as an anchor for the soul, firm and secure. It enters the inner sanctuary behind 
the curtain, where our forerunner, Jesus, has entered on our behalf. He has 
become a high priest forever, in the order of Melchizedek.
    
Heb. 7:24--28 but 
because Jesus lives forever, he has a permanent priesthood. Therefore he is able 
to save completely those who come to God through him, because he always lives 
to intercede for them. Such a high priest truly meets our need---one who is holy, 
blameless, pure, set apart from sinners, exalted above the heavens. Unlike the 
other high priests, he does not need to offer sacrifices day after day, first 
for his own sins, and then for the sins of the people. He sacrificed for their 
sins once for all when he offered himself. For the law appoints as high priests 
men in all their weakness; but the oath, which came after the law, appointed 
the Son, who has been made perfect forever.}
\newpage~Having made that sacrifice, Christ 
entered the Most Holy
Place---``heaven itself''---to appear in the presence of God on our behalf.
\sidenote{Heb. 7:25 Consequently he is able for all time to save those who 
approach God through him, since he always lives to make intercession for them.
    
Heb. 9:12 He did not enter by means of the blood of goats and calves; but he 
entered the Most Holy Place once for all by his own blood, thus obtaining 
eternal redemption.

Heb. 9:24 For Christ did not enter a sanctuary made with human hands that was 
only a copy of the true one; he entered heaven itself, now to appear for us in 
God's presence.}
He invites us to come boldly to Him, by faith, to find mercy and grace to help
us in our time of need.\sidenote{Heb 2:17--18 Therefore he had to become like 
his brothers and sisters in every respect, so that he might be a merciful 
and faithful high priest in the service of God, to make a sacrifice of 
atonement for the sins of the people. 18 Because he himself was tested by what 
he suffered, he is able to help those who are being tested.

Heb. 4:14--16 Therefore, since 
we have a great high priest who has ascended into heaven, Jesus the Son of 
God, let us hold firmly to the faith we profess. For we do not have a high 
priest who is unable to empathize with our weaknesses, but we have one who has 
been tempted in every way, just as we are---yet he did not sin. Let us then 
approach God's throne of grace with confidence, so that we may receive mercy 
and find grace to help us in our time of need.} He will soon appear, a second 
time, ``to bring salvation to those who are waiting for him.''\sidenote{Heb. 9:28
so Christ was sacrificed once to take away the sins of many; and he will appear 
a second time, not to bear sin, but to bring salvation to those who are waiting 
for him.

Heb. 10:37 For, ``In just a little while, he who is coming will come
and will not delay.''}
\end{quote}
