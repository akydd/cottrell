\chapter{Six Church Leaders who Questioned the Sanctuary Doctrine}
\label{ch:six}

\newthought{For about forty years} the sanctuary doctrine raised no known eyebrows or
protests. But on an average of every fifteen or twenty years or so since
1887 an experienced, respected, and trusted church administrator or Bible
teacher has called the attention of fellow church leaders to flaws in the
traditional interpretation of Daniel 8:14, forfeited his ministerial
credentials, and either been disfellowshiped or voluntarily left the church.
With one or two possible exceptions none of them had either spoken or taught
their doubts regarding the biblical authenticity of the sanctuary doctrine,
but were fired for thinking such thoughts and sharing them with fellow
church leaders! Furthermore, none of them were novices, but experienced
administrators or Bible teachers. Three of them had served the church
faithfully for more than half a century each.

The first church leader of record to question the sanctuary doctrine was
Dudley M. Canright, in 1887. Granted that he might have been more tactful
and patient, but for more than twenty years he had served the church as a
minister, able evangelist, administrator, and sometime member of the General
Conference Committee, and had earned the right to a fair hearing of his
views. But ``the brethren'' either did not listen or did not understand,
apparently both. He voluntarily left the church and became as bitter and
effective an opponent of Adventism as he had formerly advocated it.

Canright forthwith published a book, Seventh-day Adventism Renounced, to
warn people about the errors of Adventism. It has been translated into
scores of languages and is still used effectively to warn people against
Adventism. An honest, knowledgeable Adventist reading the book today would
have to admit that much of his tirade against the sanctuary doctrine
was---and still is---justified.\sidenote{\bibentry{20}, pages 118-126. 
For an extended discussion see~\bibentry{20a}, chapter 20}

Like Canright, Albion F. Ballenger had served the church faithfully for many
years, and in 1905 was an administrator in charge of the Irish Mission. He
was an able speaker and writer, and a diligent student of Scripture. Like 
Canright, Ballenger had never mentioned his views on the sanctuary in
public, but a committee of twenty-five the General Conference appointed to
hear him reported that he entertained views regarding the ministry of Christ
in the heavenly sanctuary contrary to that of the church. He acknowledged
the possibility that he might be wrong, and pleaded for someone to point out
from the Bible where he was wrong, but no one did, either then or later.

The church withdrew his ministerial credentials and disfellowshiped him
because of what he believed, not for anything he had said or done.
Twenty-five years later W. W. Prescott (a member of the GC ad hoc committees
appointed to meet with the dissidents) commented in a letter to W. A.
Spicer, then president of the General Conference: ``I have waited all these
years for someone to make an adequate answer to Ballenger, Fletcher and
others on their positions re\. the sanctuary but I have not seen or heard
it.'' Ballenger subsequently explained his views in the book Cast Out for the
Cross of Christ. ``No one,'' he lamented, ``who has not experienced it can
realize the soul anguish that overwhelms one who, in the study of the Word
finds truth which does not harmonize with that which he has believed and
taught during a whole lifetime to be vital to the salvation of the 
soul.''\sidenote{\bibentry{21}, pages i-iv, 1, 4, 11, 82, 106-112. See previous note.}

After some twenty years as an ordained minister, foreign missionary, and
eventually Bible teacher at Avondale College in Australia, in 1930 William
W. Fletcher voluntarily resigned from the ministry and severed his
connection with the church, under administrative pressure, solely because of
his views regarding errors in the traditional interpretation of Daniel 8:14.
Two years later he published Reasons for My Faith, setting forth his views
on the sanctuary and Christ's ministry as our great High Priest. An
objective reading of both the Bible and Reasons will conclude that
Fletcher's understanding of the former was superior to that of his
critics.\sidenote{\bibentry{22}, pages 6, 17, 23, 86, 107, 111-112, 115-138, 142-170, 220}

Louis R. Conradi served the church faithfully for fifty-two years, much of
the time as vice-president of the General Conference for the Central
European Division. He was an avid Bible scholar and student of history as
well as an able administrator, and wrote extensively. He was highly
respected by his fellow administrators. For more than thirty years questions
grew in his mind regarding the traditional interpretation of Daniel 8:14,
which he first shared with a few church leaders in 1928 and which eventually
led to a formal hearing before an ad hoc committee of thirty-three members
appointed by the General Conference, forfeiture of his ministerial
credentials, and his voluntary separation from the church in 1931. 

Thereupon he united with the Seventh Day Baptists, who issued him
ministerial credentials, gave him permission to preach Seventh-day Adventist
teachings, and made him their official representative in Europe. To his 
death he expressed confidence in the fundamental integrity of Adventism
despite errors in the sanctuary doctrine.\sidenote{See~\bibentry{20a}, chapter 20, where
where I quote extensively from original documents preserved in the General
Conference archives}

William W. Prescott was a versatile person who, over a service lifetime for
the church of more than half a century (1885-1937), distinguished himself as
a writer, editor, publisher, educator, administrator, and Bible Scholar.
Like Conradi, his study of the Bible led to a recognition of serious flaws
in the sanctuary doctrine to which, however, he never gave public
expression. He retained full confidence in the basic credibility of the
Advent message. His one ``mistake'' was in 1934 when he shared his views with
some of ``the brethren'' from headquarters, who turned against him. Unlike
Conradi, however, he remained with the church, never forfeited his
ministerial credentials, but returned to Washington, D.C. where he
fellowshipped with his critics and participated actively in various General
Conference activities.

After many years of service to the church Harold E. Snide was teaching Bible
at Southern Junior College (now Southern Adventist University). A
third-generation Adventist and a diligent student of Bible prophecy, he
encountered problems with the traditional interpretation of Daniel,
especially in connection with Christ's ministry as set forth in the book of
Hebrews. He went to the leaders in Washington with the problems that
troubled him, but found no help. The conflict between the traditional
interpretation of Daniel 8:14 and Scripture proved to be a traumatic
experience that eventually, about 1945, led him to withdraw from the church.
Mrs.\ Snide remained a loyal Adventist, however, and went to live with her
parents in Takoma Park where I became acquainted with her. 

The experience of R. A. Greive was unique in that, as president of the
Queensland Conference in Australia, he never questioned the sanctuary
doctrine. His concern was to encourage the experience of justification and
righteousness by faith as presented in the books of Romans and Hebrews, and
its counterpart the sinless perfection of Jesus Christ. Church leaders in
the division office, however, accused him of thereby being in conflict with
the concept of an investigative judgment as the cleansing of the sanctuary
referred to in Daniel 8:14 and explained in Hebrews 9.

If, as Paul wrote in Romans 8:1, there is ``now no condemnation for those who
are in Christ Jesus,'' how can a record of those sins be preserved and
reviewed during the course of an investigative judgment? Greive asked. He
also pointed out that, according to Hebrews 7:27 and 9:6-12, Christ
completed His equivalent of the first apartment ministry on the cross and
entered upon His equivalent of the second apartment ministry when He
ascended to heaven, not eighteen centuries later. At his trial Greive agreed
to go as far as his ``enlightened conscience'' would allow in order to have 
harmony with his brethren, but for them that was not far enough. In 1956 his
credentials were withdrawn and he withdrew from the church.\sidenote{For detailed
information regarding R. A. Greive, see~\bibentry{24}, pages 89-95, or \bibentry{24a},
pages 55-61}

Think of the time, attention, and cost of disciplining these six
administrators and Bible scholars, listed above, has diverted from the
mission of the church to the world! Think also of the distress and heartache
these six have experienced and often expressed. Think, as well, of the
damage some of them have done to the church! 
