\chapter{Non-Adventist Reaction to the Sanctuary Doctrine}
\label{ch:non-sda}

\newthought{It was the sanctuary doctrine} based on Daniel 8:14 that made us Seventh-day
Adventists and that remains, today, the keystone of our distinctive belief
system and our mission to the world. Of it, Ellen White wrote: ``The
Scripture which above all others had been both the foundation and central 
pillar of our faith was the declaration, `Unto two thousand and three
hundred days; then shall the sanctuary be cleansed'\,''\sidenote{\bibentry{8},
page 409} and ``The correct
understanding of the ministration in the heavenly sanctuary is the
foundation of our faith.'' ``Not one pin is to be removed from that which the
Lord has established. The enemy will bring in false theories, such as the
doctrine that there is no sanctuary. This is one of the points on which
there will be a departing from the faith.''\sidenote{\bibentry{31}, pages 221, 224}

When, in the mid-1950's, Walter Martin and Donald Grey Barnhouse explored
Adventist teachings in depth with persons appointed by the General
Conference, they concluded that, with two exceptions, we are in harmony with
the gospel:
\begin{enumerate}
    \item our sanctuary doctrine, and 
    \item the role we popularly
ascribe to Ellen White as an infallible interpreter of Scripture, in
contradiction of her own explicit statements to the contrary.
\end{enumerate}
The former,
they concluded, violates the Reformation principle sola
Scriptura.\sidenote{My 28-page paper \bibentry{32a} is a detailed review of the
eighteen Martin-Barnhouse interviews with General Conference personnel in 1955
and 1956.  My 10-page \bibentry{32b} recounts a number of humorous moments during
the Martin-Barnhouse interviews.} Of it, Barnhouse wrote: 
\begin{quote}
The [sanctuary] doctrine is, to me, the most colossal,
psychological, face-saving phenomenon in religious history. \ldots We
personally do not believe that there is even a suspicion of a verse in
Scripture to sustain such a peculiar position, and we further believe that
any effort to establish it is stale, flat, and unprofitable. \ldots [It is]
unimportant and almost na\"{\i}ve.\cite{33} 
\end{quote}

\newpage
Such is the usual reaction of non-Adventist Bible scholars and other
biblically literate non-Adventists to our sanctuary doctrine.\sidenote{My
16-page \bibentry{34} quotes from, and summarizes, comment on the contemporary
(1956) Evangelical Christian press regarding the Martin-Barnhouse interviews.
This document was prepared as the request of the editorial committee preparing
\bibentry{34a} for publication.}
