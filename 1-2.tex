\chapter{Ellen G. White and the Sanctuary Doctrine}
\label{ch:egw}

\newthought{The ultimate argument} in defense of the traditional interpretation of Daniel
8:14 every time questions have been raised concerning it, has been Ellen
White's explicit affirmation of it. As a presumably infallible interpreter
of Scripture her support always settled the matter. For instance, in 1888,
forty-four years after the great disappointment of October 22, 1844 she
wrote: ``The scripture which above all others had been both the foundation
and the central pillar of the advent faith, was the declaration, `Unto two
thousand and three hundred days; then shall the sanctuary be 
cleansed'.''\sidenote{\bibentry{8}, page 409}

She devoted an entire chapter in The Great Controversy to a defense and
explanation of the sanctuary doctrine.\sidenote{\bibentry{8}, pages 409-422}
 Eighteen years later, in 1906, she
 wrote again: ``The correct understanding of the ministration in the heavenly
 sanctuary is the foundation of our faith.''\sidenote{\bibentry{10}, page 221}

In order to understand these two statements in their historical context it
is important to remember that she and many others then living had personally
experienced the great disappointment of October 22, 1844. Her statements
about it were absolutely historically accurate. The experience was still
vivid in her own mind and in the minds of many others. 

\newpage
In both of these statements Ellen White is simply stating historical fact;
she is not exegeting Scripture. In 1895 she wrote: ``In regard to
infallibility, I never claimed it; God alone is infallible.''\cite{11} ``The Bible is
the only rule of faith and doctrine. \ldots The Bible alone \ldots [is] the
foundation of our faith. \ldots The Bible alone is to be our guide. The Holy
Scriptures are to be accepted as an authoritative, infallible revelation of
[God's] will. \ldots We are to receive God's word as supreme authority.''
\sidenote{\bibentry{12}, pages 112, 126. \bibentry{12a}, page 21. \bibentry{12b}, 
page 85. \bibentry{12c}, page 145. \bibentry{12d}, pages 663, 691. \bibentry{12e}, 
page 402. \bibentry{8}, page vii. \bibentry{12g}, page 125}
Numerous similar statements could be cited.\sidenote{\bibentry{12a}, pages 37, 164.
\bibentry{13b}, page 33} It is important to remember
that she never considered herself an exegete of the Bible. Upon numerous
occasions when asked for what her questioners proposed to accept as an
authoritative, infallible interpretation of a disputed Bible passage she
refused, and told them to go to the Bible themselves for an answer. 

It is also vital to remember that in Ellen White's 47,000\sidenote{\bibentry{14},
pages 21-176. An
estimate of the entries} or so citations
of Scripture she makes use of the Bible in two distinct ways:
\begin{enumerate}
        \item to quote the Bible when narrating the Bible story in its own
            context, and
        \item to apply Bible principles in her counsel to the church today---out
            of its biblical context. 
\end{enumerate}
\newpage
A clear illustration of this two-fold use of the Bible is her series of
comments on Galatians 3:24: ``The law was our schoolmaster to bring us to
Christ.''
\begin{enumerate}
        \item In 1856 she identified that law as the ceremonial law system of
ancient times, and specifically not the Ten Commandments.\sidenote{\bibentry{15a},
\bibentry{15b}, pages 70, 108.  In 1856 James and Ellen White and others met for
two days in Battle
Creek, Michigan, and decided that Waggoner was wrong in identifying the law
in Galatians as the Ten Commandments. James White withdrew the book from
circulation.}
        \item In 1883 she
        again identified that ``law'' as ``the obsolete ceremonies of 
        Judaism.''\sidenote{\bibentry{16}, pages 188-192}
        \item In 1896 she wrote: ``In this Scripture, the Holy Spirit through the apostle
            is speaking especially of the moral law.''\sidenote{\bibentry{12a}, page 234}
        \item In 1900 she wrote: ``I am
asked concerning the law in Galatians. \ldots I answer: both the ceremonial and
        moral code of Ten Commandments.''\sidenote{\bibentry{12a}, page 233}
    \item In 1911 she again identified the law
        in Galatians as exclusively ``the obsolete ceremonies of 
        Judaism.''\sidenote{\bibentry{19}, pages 383-388}
\end{enumerate}

In these three reversals (ceremonial law exclusively, Ten Commandments
exclusively, both the ceremonial law and the Ten Commandments, ceremonial
law exclusively) was she contradicting herself or did she repeatedly change
her mind? Neither! A careful reading of each statement in its own context
makes evident that
\begin{enumerate}
        \item when she identifies the law in Galatians as the
ceremonial law system of ancient times she is commenting on Galatians in its
own historical context, and 
        \item when she applies the principle involved to
our time she does so out of its biblical context.
\end{enumerate}
The principle involved in
Paul's day and in ours is identical: the Galatians could not be saved by a
rigorous observance of the ceremonial laws; nor can we be saved by a
rigorous observance of the Ten Commandments! The two contradictory
definitions of the law in Galatians are both valid and accurate! A careful
examination of Ellen White's thousands of quotations from, or allusion to,
the Bible makes evident that her historical statements regarding Daniel 8:14
are historically accurate with respect to the 1844 experience and not a
denial of what the passage meant in Daniel's time. 

We may think of the heavenly sanctuary explanation of the great
disappointment as a prosthetic device, a spiritual crutch that enabled the
``little flock'' of Adventist pioneers ``scattered abroad'' to survive the
great disappointment of October 22, 1844 and not lose faith in the imminent
return of Jesus, as so many others did. That explanation was the best they
could do, given the prooftext method on which, of necessity, they relied.
With the historical method at our disposal today, we no longer need that
crutch and would do well to lay it up on the shelf of history. It is
counterproductive in our witness to the everlasting gospel today, to
biblically literate Adventists and non-Adventists alike. 
