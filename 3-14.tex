\chapter{The Authenticity of Adventism}
\label{ch:authenticity}

This review and analysis of the traditional Adventist interpretation of
Daniel 8:14, the sanctuary, and the investigative judgment is designed to be
constructive and remedial, not critical, accusatory, or punitive. I
sincerely hope that it will be received in the same spirit, and that
appropriate action will be taken to spare the church and its members from a
repetition of the traumatic episodes of the past for which this
pseudo-biblical doctrine, historicism, and obscurantism have been
responsible.

For two reasons Seventh-day Adventism remains an authentic, credible witness
to the everlasting gospel despite its all-to-human imperfections such as its
traditional interpretation of Daniel 8:14, the sanctuary doctrine, and the
investigative judgment:
\begin{enumerate}
\item Its unique emphasis on applying the gospel of
Jesus Christ to every aspect of human personhood, mental and physical as
well as spiritual and social---practical, loving concern for the well-being
and happiness of all human beings, and 
\item its emphatic witness to His
promised, imminent return to transform this suffering little world into the
permanent abode of righteousness and peace He originally designed it to
enjoy.
\end{enumerate}

In view of the fact that Seventh-day Adventists have, historically and
today, relied on the authenticity of the 1844 experience and the basic
credibility of the traditional interpretation of Daniel 8:14, and in view of
the above evidence that that interpretation is not tenable when tested by
the sola Scriptura principle (which the church affirms but compromises in
its interpretation of Daniel 8:14), the question inevitably arises, ``What
basis is there for concluding that Adventism is an authentic witness to the
everlasting gospel of Jesus Christ?'' An inevitable and appropriate question
indeed!

The pragmatic response to that question is the extent to which the church
conforms to, and reflects, the teachings of Jesus Christ and complies with
the gospel commission. Whether or not it does so uniquely is none of our
business or concern. Even to be concerned with that question violates His
specific instruction on record in Mark 9:38--41. Someone was casting out
demons in Jesus' name and the disciples ``tried to stop him, because he was
not following us. But Jesus said, `Do not stop him; for no one who does a
deed of power in my name will be able soon afterward to speak evil of me.
Whoever is not against us is for us.'\,'' On another occasion Peter, pointing
to John, asked Jesus ``What about him?'' In His reply Jesus said to Peter,
``What is that to you? Follow me.'' It is none of our business as Seventh-day
Adventists to question the credibility or integrity of others as authentic
witnesses of Jesus Christ. Let us focus our attention on the credibility of
our witness to the everlasting gospel---and banish any ``holier than thou''
questions from our minds. In Acts 10:35 Peter says, ``In every nation [and
religious community] anyone who fears him and does what is right is
acceptable to him.''

Jesus' summary of the gospel is on record in Mark 12:29--31: ``You shall love
the Lord your God with all your heart, and with all your soul, and with all
your mind, and with all your strength,'' and ``You shall love your neighbor as
yourself.'' This is the true test of us corporately as a church as well as of
us individually, as members of the church. In other words, gospel principles
apply to every aspect of our individual and corporate being---our love for,
and the dedication of our entire individual and corporate being, to 
God---and in our relationship to one another and to every other human being.
``As you did it to one of the least of these my brethren, you did it to
me.''\sidenote{Matthew 25:40 And the king will answer them, `Truly I tell you,
just as you did it to one of the least of these who are members of my family,
you did it to me.'} The agape love of God is selfless concern and care for the
well-being and happiness of others. That must be the ideal and practice of
the church with respect to every human being everywhere, in theory but even
more importantly, in practice. ``In as much\ldots''!

We are admitted to eternity on the basis of the kind of people we are,
individually, not what we may sincerely believe about Daniel 8:14 or any
other passage of Scripture. A person may conscientiously believe in the
traditional interpretation of Daniel 8:14, and if everything else in his or
her life is in harmony with the gospel he/she will encounter no problem at
the pearly gates of eternity. And if a person sincerely believes that is not
its import, but everything else in his/her life is in harmony with the
gospel, he/she will encounter no problem at the pearly gates of eternity.
But is we become abusive of one another in our discussion of the subject we
will both arrive at the pearly gates only to find them bolted and barred
against both of us.

Let our corporate attitude as a church be in moderated by this fact, but at
the same time let the church, corporately, be in full harmony with the sola
Scriptura principle in its delineation of, and witness to, Daniel 8:14. In
terms of sola Scriptura its sanctuary witness to the gospel is grossly
defective and alienates the confidence and respect of biblically literate
people, Adventist and non-Adventist alike. Let us be willing to recognize
and remove that obstacle to acceptance of our message to the world that
Jesus will soon return.

In the years immediately following October 22, 1844 the traditional
sanctuary doctrine was an important asset for stabilizing the faith of
disappointed Adventists. Today it is an equally significant liability and
deterrent to the faith, confidence, and salvation of biblically literate
Adventists and non-Adventists alike. It was present truth following the
great disappointment on October 22, 1844. It is not present truth in the
year of our Lord 2002. \textit{Quod erat demonstrandum!}

\vspace{1 cm}

Raymond F. Cottrell, \\ 335 Midori Lane, Calimesa, CA 92320-1615 \\ February 9, 2002 \\ r.rc@gte.net
