\part*{Introduction}

The traditional interpretation of Daniel 8:14 with its sanctuary and investigative
judgment, which gave birth to Seventh-day Adventism and accounts for its
existence as a distinct entity within Christendom, has been the object of more
criticism and debate, by both Adventists and non-Adventists, than all other facets
of its belief system combined. The same is true with respect to church discipline
on doctrinal grounds, defections from the church, and the diversion of time,
attention, and resources from Adventism's perceived mission to the world.

It has been repeatedly and consistently demonstrated that an ordained minister
may believe that Christ was a created being (and not God in the full sense of the
word), or that a person can earn salvation by faithfully observing the Ten
Commandments, or that Genesis 1 is not a literal account of creation a mere six
thousand years ago---without being disciplined and forfeiting his ministerial
credentials. But it has also been repeatedly and consistently demonstrated that
an ordained minister may not conscientiously question the authenticity of the
traditional interpretation of Daniel 8:14, even in his thoughts, without his
ministerial credentials being revoked. As noted below, in several instances as
much as half a century of faithful service to the church has not been sufficient to
mitigate this result.

Accordingly, it is appropriate to review the origin, history, and methodology of
the sanctuary doctrine, to examine it on the basis of the sola Scriptura principle
and recognized principles of exegesis, and to explore procedures by means of
which to avoid repeating the traumatic experiences of the church with it in the
past---to learn from experience.

Insofar as possible this paper avoids technical hermeneutical terminology,
including the transliteration of Hebrew words used by Bible scholars. The
transliteration used is designed to enable persons not familiar with biblical
Hebrew to approximate the Hebrew vocalization. Except as otherwise noted, Bible
quotations cited are from the New Revised Standard Version (NRSV).
