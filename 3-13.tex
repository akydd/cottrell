\chapter{A Permanent Remedy for Obscurantism}
\label{ch:remedy}

The church urgently needs a bona fide consensus of all of its qualified
Bible scholars in order to ascertain as accurately as possible all matters
of biblical exegesis in harmony with the sola Scriptura principle, 
preliminary to the formulation of doctrinal statements in partnership with
church administrators. Such a consensus can be achieved only by an
organization that would provide its members with an opportunity to confer
together apart from every influence or concern other than faithfulness to
sola Scriptura and loyalty to the church.

\begin{enumerate}
\item This organization would serve as an agency of, funded by, and dedicated
to cooperating with the General Conference, with the specific objective of
providing GC administrators with a bona fide consensus of its community of
Bible scholars on all biblical and doctrinal matters. 
\item It would
participate with the GC in defining their working relationship. 
\item It would
select its name (for example, ``Bible Scholars' Council on Biblical
Exegesis''). 
\item It would define its membership requirements, 
\item select its
officers and specify their terms of service, and 
\item elect an executive
committee and a permanent staff. 
\item It would define its operating
procedures, 
\item set its own agenda, 
\item receive and respond to requests from
the GC, 
\item select topics of its own for consideration, and 
\item define its
principles of exegesis.
\item It would report to GC administration only, and not otherwise publicize
its findings beyond scholarly circles. 
\item Its reports to administration
would reflect both the majority consensus and the degree of minority
dissent, if any. 
\item It would conduct most of its business via e-mail, but
\item hold an annual convocation which all members would be invited to
attend, with their employing organizations funding travel and accomodations.
\item It would ordinarily meet in camera, but might, at its discretion,
invite non-scholar observers. 
\item Its formative stage might be limited to
North American Bible scholars, but eventually it should include all
qualified Adventist Bible scholars worldwide.
\end{enumerate}

Such an organization would be of inestimable value to the church. It would
help the church to be a faithful witness to the sola Scriptura principle in
all aspects of its witness to the everlasting gospel, and to avoid the
obscurantism and intermittent doctrinal controversy of the past century.
