\part{Origin and History of the Sanctuary Doctrine}
\label{p:origin}

\chapter{Formation of the Sanctuary Doctrine}
\label{ch:formation}

\newthought{Pioneer Seventh-day Adventists} inherited their identification of the year
1844 as the terminus of the 2300 ``days'' foretold in the KJV of Daniel 8:14
from William Miller. Formerly an avowed skeptic, he was converted in 1816
and eventually became a Baptist lay preacher. He devoted his first two years 
as a born-again Christian to a diligent study of the Bible, which eventually
came to a focus on Daniel 8:14 and the conclusion that it foretold the
second coming of Christ ``about the year 1843.''

According to the Seventh-day Adventist Encyclopedia Miller ``repeatedly
declared that his prophetic views were not new'', but insisted that he came
to his conclusions exclusively through his own study of the Bible and
reference to a concordance. In volume 4 of his Prophetic Faith of Our
Fathers Le Roy Edwin Froom notes that Miller was by no means the
``originator'' of the idea that the 2300 ``days'' were prophetic years ending
about 1843, and that it is ``a simple historical fact that the origin of the
view of the 2,300 years as ending at that time, and its wide circulation,
was wholly prior to and independent of William Miller.''\cite{1}

By what process did Miller, this formidable array of Bible students, and
pioneer Adventists arrive at 1843/44 as the terminus of the 2300 ``days'' of
Daniel 8:14? Relying on the 1611 King James translation of the Bible (the
only one then available), they
\begin{enumerate}
    \item identified its ``sanctuary'' as the church on earth
    \item accepted the KJV interpretation of erev boquer (literally, ``evening
        morning'') as ``days''
    \item adopted the ``day-for-a-year'' principle in Bible prophecy and thus
        construed the 2300 ``days'' as prophetic years
    \item took the seventy ``weeks'' of Daniel 9:24-27 as the first segment of
        these 2300 years
    \item identified the cessation of sacrifice and offering for the last half
        of the seventieth of the seventy ``weeks'' (verse 27) as referring to
        Jesus' crucifixion\footnote{Matt. 57:21 At that moment the curtain of
        the temple was torn in two from top to bottom. The earth shook, the
        rocks split.}
    \item figuring back from the crucifixion, they identified the decree of the
        Persian king Artaxerxes Longimanus in his seventh year (Ezra 7) as that
        alluded to in Daniel 9:25, thus locating the commencement of the 2300
        years in 457 B.C.
    \item with 457 B.C. as their starting point, terminated them ``about the
        year 1843''
    \item adopted the KJV interpretation of nitsdaq (literally, ``set right''
        or ``restored'') as ``cleansed'', and 
    \item concluded that the cleansing of the sanctuary of Daniel 8:14
        meant the cleansing of the church on earth (and thus the earth
        itself) by fire at the second coming of Christ.
\end{enumerate}

When the great disappointment of October 22, 1844 proved conclusively that
Miller's identification of the ``sanctuary'' in Daniel 8:14 as the church on
earth and the nature of its cleansing as by fire at the second coming of
Christ\sidenote{2 Peter 3:7--12 By the same word the present heavens and earth
are reserved for fire, being kept for the day of judgment and destruction of the
ungodly. But do not forget this one thing, dear friends: With the Lord a day is like a
thousand years, and a thousand years are like a day. The Lord is not slow in
keeping his promise, as some understand slowness. Instead he is patient with
you, not wanting anyone to perish, but everyone to come to repentance.
But the day of the Lord will come like a thief. The heavens will disappear
with a roar; the elements will be destroyed by fire, and the earth and everything
done in it will be laid bare.
Since everything will be destroyed in this way, what kind of people ought
you to be? You ought to live holy and godly lives as you look forward to the
day of God and speed its coming. That day will bring about the destruction of
the heavens by fire, and the elements will melt in the heat.},
 were in error, pioneer Adventists re-identified the ``sanctuary'' of
verse 14 as that of the Book of Hebrews in heaven\sidenote{Heb. 8:2 and who
serves in the sanctuary, the true tabernacle set up by the Lord, not by a mere
human being.}, and its cleansing as the heavenly counterpart of the cleansing
of the ancient sanctuary on the Day of Atonement\sidenote{Lev. 16}.

Retaining, however, the presumed validity of October 22, 1844 as the
fulfillment of Daniel 8:14 and the concept that it implied the soon return
of their Lord, the disappointed Adventist pioneers assumed that human
probation had indeed closed on that fateful day, and that only those who at 
that time awaited His return were eligible for eternal life. They referred
to this concept as ``the shut door'' in the parable of the Ten
Virgins\sidenote{Matt. 25:1--13}.
 They
soon mated the ``shut door'' theory to the idea that the sanctuary of Daniel
8:14 was the sanctuary in heaven, of the book of Hebrews, that the ``shut
door'' was the ``door'' between its holy and most holy apartments, that on
October 22 Christ had closed His ministry in the holy place and entered upon
His high priestly ministry in its most holy place, and referred to His
ministry there as an ``investigative judgment''.

For several years the ``little flock'' of pioneer Seventh-day Adventists
``scattered abroad'' believed that the investigative judgment phase of
Christ's ministry would be very brief (not more than five years or so at the
most)\cite{7},
 following which He would immediately return to earth. The eventual
accession of new, non-1844, members to the ``little flock'' proved to be
convincing evidence that the door of mercy remained open, and by the early
1850's they abandoned the ``shut door'' aspect of the sanctuary-in-heaven
interpretation of Daniel 8:14.

This completed the traditional Adventist interpretation of Daniel 8:14, the
sanctuary, and the investigative judgment, which was thereafter commonly
referred to as ``the sanctuary doctrine'' set forth in every statement of
beliefs, most recently as article 23 of the 27 Fundamental Beliefs adopted
at the 1980 session of the General Conference in New Orleans.

\chapter{Ellen G. White and the Sanctuary Doctrine}
\label{ch:egw}

\newthought{The ultimate argument} in defense of the traditional interpretation of Daniel
8:14 every time questions have been raised concerning it, has been Ellen
White's explicit affirmation of it. As a presumably infallible interpreter
of Scripture her support always settled the matter. For instance, in 1888,
forty-four years after the great disappointment of October 22, 1844 she
wrote: ``The scripture which above all others had been both the foundation
and the central pillar of the advent faith, was the declaration, `Unto two
thousand and three hundred days; then shall the sanctuary be cleansed'.''\cite{8}

She devoted an entire chapter in The Great Controversy to a defense and
explanation of the sanctuary doctrine.\cite{9}
 Eighteen years later, in 1906, she
 wrote again: ``The correct understanding of the ministration in the heavenly
 sanctuary is the foundation of our faith.''\cite{10}

In order to understand these two statements in their historical context it
is important to remember that she and many others then living had personally
experienced the great disappointment of October 22, 1844. Her statements
about it were absolutely historically accurate. The experience was still
vivid in her own mind and in the minds of many others. 

\newpage
In both of these statements Ellen White is simply stating historical fact;
she is not exegeting Scripture. In 1895 she wrote: ``In regard to
infallibility, I never claimed it; God alone is infallible.''\cite{11} ``The Bible is
the only rule of faith and doctrine. \ldots The Bible alone \ldots [is] the
foundation of our faith. \ldots The Bible alone is to be our guide. The Holy
Scriptures are to be accepted as an authoritative, infallible revelation of
[God's] will. \ldots We are to receive God's word as supreme authority.''
\cite{12, 12a, 12b, 12c, 12d, 12e, 12f, 12g}
Numerous similar statements could be cited.\cite{13a, 13b} It is important to remember
that she never considered herself an exegete of the Bible. Upon numerous
occasions when asked for what her questioners proposed to accept as an
authoritative, infallible interpretation of a disputed Bible passage she
refused, and told them to go to the Bible themselves for an answer. 

It is also vital to remember that in Ellen White's 47,000\sidenote{\bibentry{14}. An
estimate of the entries} or so citations
of Scripture she makes use of the Bible in two distinct ways:
\begin{enumerate}
        \item to quote the Bible when narrating the Bible story in its own
            context, and
        \item to apply Bible principles in her counsel to the church today---out
            of its biblical context. 
\end{enumerate}
\newpage
A clear illustration of this two-fold use of the Bible is her series of
comments on Galatians 3:24: ``The law was our schoolmaster to bring us to
Christ.''
\begin{enumerate}
        \item In 1856 she identified that law as the ceremonial law system of
ancient times, and specifically not the Ten Commandments.\sidenote{\bibentry{15a},
\bibentry{15b}.  In 1856 James and Ellen White and others met for two days in Battle
Creek, Michigan, and decided that Waggoner was wrong in identifying the law
in Galatians as the Ten Commandments. James White withdrew the book from
circulation.}
        \item In 1883 she
        again identified that ``law'' as ``the obsolete ceremonies of Judaism.''\cite{16}
        \item In 1896 she wrote: ``In this Scripture, the Holy Spirit through the apostle
            is speaking especially of the moral law.''\cite{17}
        \item In 1900 she wrote: ``I am
asked concerning the law in Galatians. \ldots I answer: both the ceremonial and
        moral code of Ten Commandments.''\cite{18} 
    \item In 1911 she again identified the law
        in Galatians as exclusively ``the obsolete ceremonies of Judaism.''\cite{19}
\end{enumerate}

In these three reversals (ceremonial law exclusively, Ten Commandments
exclusively, both the ceremonial law and the Ten Commandments, ceremonial
law exclusively) was she contradicting herself or did she repeatedly change
her mind? Neither! A careful reading of each statement in its own context
makes evident that
\begin{enumerate}
        \item when she identifies the law in Galatians as the
ceremonial law system of ancient times she is commenting on Galatians in its
own historical context, and 
        \item when she applies the principle involved to
our time she does so out of its biblical context.
\end{enumerate}
The principle involved in
Paul's day and in ours is identical: the Galatians could not be saved by a
rigorous observance of the ceremonial laws; nor can we be saved by a
rigorous observance of the Ten Commandments! The two contradictory
definitions of the law in Galatians are both valid and accurate! A careful
examination of Ellen White's thousands of quotations from, or allusion to,
the Bible makes evident that her historical statements regarding Daniel 8:14
are historically accurate with respect to the 1844 experience and not a
denial of what the passage meant in Daniel's time. 

We may think of the heavenly sanctuary explanation of the great
disappointment as a prosthetic device, a spiritual crutch that enabled the
``little flock'' of Adventist pioneers ``scattered abroad'' to survive the
great disappointment of October 22, 1844 and not lose faith in the imminent
return of Jesus, as so many others did. That explanation was the best they
could do, given the prooftext method on which, of necessity, they relied.
With the historical method at our disposal today, we no longer need that
crutch and would do well to lay it up on the shelf of history. It is
counterproductive in our witness to the everlasting gospel today, to
biblically literate Adventists and non-Adventists alike. 

\chapter{Six Church Leaders who Questioned the Sanctuary Doctrine}
\label{ch:six}

\newthought{For about forty years} the sanctuary doctrine raised no known eyebrows or
protests. But on an average of every fifteen or twenty years or so since
1887 an experienced, respected, and trusted church administrator or Bible
teacher has called the attention of fellow church leaders to flaws in the
traditional interpretation of Daniel 8:14, forfeited his ministerial
credentials, and either been disfellowshiped or voluntarily left the church.
With one or two possible exceptions none of them had either spoken or taught
their doubts regarding the biblical authenticity of the sanctuary doctrine,
but were fired for thinking such thoughts and sharing them with fellow
church leaders! Furthermore, none of them were novices, but experienced
administrators or Bible teachers. Three of them had served the church
faithfully for more than half a century each.

The first church leader of record to question the sanctuary doctrine was
Dudley M. Canright, in 1887. Granted that he might have been more tactful
and patient, but for more than twenty years he had served the church as a
minister, able evangelist, administrator, and sometime member of the General
Conference Committee, and had earned the right to a fair hearing of his
views. But ``the brethren'' either did not listen or did not understand,
apparently both. He voluntarily left the church and became as bitter and
effective an opponent of Adventism as he had formerly advocated it.

Canright forthwith published a book, Seventh-day Adventism Renounced, to
warn people about the errors of Adventism. It has been translated into
scores of languages and is still used effectively to warn people against
Adventism. An honest, knowledgeable Adventist reading the book today would
have to admit that much of his tirade against the sanctuary doctrine
was---and still is---justified.\sidenote{\bibentry{20}. For an extended discussion
see~\bibentry{20a}}

Like Canright, Albion F. Ballenger had served the church faithfully for many
years, and in 1905 was an administrator in charge of the Irish Mission. He
was an able speaker and writer, and a diligent student of Scripture. Like 
Canright, Ballenger had never mentioned his views on the sanctuary in
public, but a committee of twenty-five the General Conference appointed to
hear him reported that he entertained views regarding the ministry of Christ
in the heavenly sanctuary contrary to that of the church. He acknowledged
the possibility that he might be wrong, and pleaded for someone to point out
from the Bible where he was wrong, but no one did, either then or later.

The church withdrew his ministerial credentials and disfellowshiped him
because of what he believed, not for anything he had said or done.
Twenty-five years later W. W. Prescott (a member of the GC ad hoc committees
appointed to meet with the dissidents) commented in a letter to W. A.
Spicer, then president of the General Conference: ``I have waited all these
years for someone to make an adequate answer to Ballenger, Fletcher and
others on their positions re\. the sanctuary but I have not seen or heard
it.'' Ballenger subsequently explained his views in the book Cast Out for the
Cross of Christ. ``No one,'' he lamented, ``who has not experienced it can
realize the soul anguish that overwhelms one who, in the study of the Word
finds truth which does not harmonize with that which he has believed and
taught during a whole lifetime to be vital to the salvation of the soul.''\cite{21, 20a}

After some twenty years as an ordained minister, foreign missionary, and
eventually Bible teacher at Avondale College in Australia, in 1930 William
W. Fletcher voluntarily resigned from the ministry and severed his
connection with the church, under administrative pressure, solely because of
his views regarding errors in the traditional interpretation of Daniel 8:14.
Two years later he published Reasons for My Faith, setting forth his views
on the sanctuary and Christ's ministry as our great High Priest. An
objective reading of both the Bible and Reasons will conclude that
Fletcher's understanding of the former was superior to that of his
critics.\cite{22}

Louis R. Conradi served the church faithfully for fifty-two years, much of
the time as vice-president of the General Conference for the Central
European Division. He was an avid Bible scholar and student of history as
well as an able administrator, and wrote extensively. He was highly
respected by his fellow administrators. For more than thirty years questions
grew in his mind regarding the traditional interpretation of Daniel 8:14,
which he first shared with a few church leaders in 1928 and which eventually
led to a formal hearing before an ad hoc committee of thirty-three members
appointed by the General Conference, forfeiture of his ministerial
credentials, and his voluntary separation from the church in 1931. 

Thereupon he united with the Seventh Day Baptists, who issued him
ministerial credentials, gave him permission to preach Seventh-day Adventist
teachings, and made him their official representative in Europe. To his 
death he expressed confidence in the fundamental integrity of Adventism
despite errors in the sanctuary doctrine.\sidenote{See~\bibentry{20a}, where
where I quote extensively from original documents preserved in the General
Conference archives}

William W. Prescott was a versatile person who, over a service lifetime for
the church of more than half a century (1885-1937), distinguished himself as
a writer, editor, publisher, educator, administrator, and Bible Scholar.
Like Conradi, his study of the Bible led to a recognition of serious flaws
in the sanctuary doctrine to which, however, he never gave public
expression. He retained full confidence in the basic credibility of the
Advent message. His one ``mistake'' was in 1934 when he shared his views with
some of ``the brethren'' from headquarters, who turned against him. Unlike
Conradi, however, he remained with the church, never forfeited his
ministerial credentials, but returned to Washington, D.C. where he
fellowshipped with his critics and participated actively in various General
Conference activities.

After many years of service to the church Harold E. Snide was teaching Bible
at Southern Junior College (now Southern Adventist University). A
third-generation Adventist and a diligent student of Bible prophecy, he
encountered problems with the traditional interpretation of Daniel,
especially in connection with Christ's ministry as set forth in the book of
Hebrews. He went to the leaders in Washington with the problems that
troubled him, but found no help. The conflict between the traditional
interpretation of Daniel 8:14 and Scripture proved to be a traumatic
experience that eventually, about 1945, led him to withdraw from the church.
Mrs.\ Snide remained a loyal Adventist, however, and went to live with her
parents in Takoma Park where I became acquainted with her. 

The experience of R. A. Greive was unique in that, as president of the
Queensland Conference in Australia, he never questioned the sanctuary
doctrine. His concern was to encourage the experience of justification and
righteousness by faith as presented in the books of Romans and Hebrews, and
its counterpart the sinless perfection of Jesus Christ. Church leaders in
the division office, however, accused him of thereby being in conflict with
the concept of an investigative judgment as the cleansing of the sanctuary
referred to in Daniel 8:14 and explained in Hebrews 9.

If, as Paul wrote in Romans 8:1, there is ``now no condemnation for those who
are in Christ Jesus,'' how can a record of those sins be preserved and
reviewed during the course of an investigative judgment? Greive asked. He
also pointed out that, according to Hebrews 7:27 and 9:6-12, Christ
completed His equivalent of the first apartment ministry on the cross and
entered upon His equivalent of the second apartment ministry when He
ascended to heaven, not eighteen centuries later. At his trial Greive agreed
to go as far as his ``enlightened conscience'' would allow in order to have 
harmony with his brethren, but for them that was not far enough. In 1956 his
credentials were withdrawn and he withdrew from the church.\sidenote{For detailed
information regarding R. A. Greive, see~\bibentry{24} or \bibentry{24a}}

Think of the time, attention, and cost of disciplining these six
administrators and Bible scholars, listed above, has diverted from the
mission of the church to the world! Think also of the distress and heartache
these six have experienced and often expressed. Think, as well, of the
damage some of them have done to the church! 

\chapter{Continuing Casualties of the Sanctuary Doctrine}
\label{ch:cont}

\newthought{Like an airplane} unexpectedly entering a region of clear air turbulence, in
1945 Dr.\ Desmond Ford began to encounter exegetical problems in the
traditional Adventist interpretation of Daniel 8:14, the sanctuary, and the
investigative judgment. He set out to put all of the disparate pieces
together in a coherent pattern that would resolve the problems, that would
be faithful to reliable principles of exegesis, and that left him a
dedicated Seventh-day Adventist with complete confidence in the integrity of
the church as an authentic witness to the everlasting gospel.

Over the next ten or fifteen years Ford discovered that some of his
contemporaries and others before him had wrestled with the same problems. In
his definitive 991-page Glacier View document, Daniel 8:14, the Day of
Atonement, and the Investigative Judgment, he names twelve Adventist Leaders
with whom he had discussed the problems, in person or by correspondence. He
devoted his master's and one of his doctoral dissertations to the subject.
His published commentaries on the Books of Daniel and the Revelation total
more than two thousand pages. He has probably devoted more scholarly study
to the subject and written more extensively on it than any other person in
history.

During his long tenure as head of the theology department at Avondale
College in Australia he trained half or so of the ministers in Australia. In
the classroom and by his personal example he inspired thousands of young
people for Christ. He was always in demand as a speaker, and thousands
testify to a clearer understanding and appreciation of the gospel as a
result of his witness to it. His theme ever was---and still is---salvation
by faith in Jesus Christ.

Ford never discussed the controversial aspects of the sanctuary doctrine in
public---until October 27, 1979, as an exchange professor at Pacific Union
College, when several members of the faculty invited him to discuss his
views on the sanctuary question in an open meeting one Sabbath afternoon.
Thirty-four years of silence on the subject surely reflect commendable
pastoral and scholarly restraint. The PUC presentation ``was positive on the 
providential role of Adventists and Ellen White.'' However, three retired
ministers present detected what they perceived to be heresy and reported
their version of his remarks to the chairman of the college board. 

In view of the fact that Ford was still an employee of Avondale College in
Australia and due to return to Avondale at the close of the 1979-1980 school
year, the chairman logically referred the matter to the General Conference.
In August 1980 115 leading administrators and Bible scholars from around the
world (at an administrator's estimated cost of a quarter of a million
dollars) were summoned to Glacier View\sidenote{For a summary of highlights
of the 991-page \bibentry{25a}, 
see my
18-page paper, \bibentry{25b}. For a very
detailed account of proceedings at the Glacier View meeting of the Sanctuary
Review Committee, August 10-15, 1980, see my report \bibentry{25c}.
This article is based on my complete shorthand notes of every speech and all
proceedings at the morning Study Group 2, of which I was a member, and the
afternoon and evening plenary sessions. My unpublished 20-page paper \bibentry{25d}
explains what happened at Glacier View and why it
happened as it did. My 21-page unpublished paper \bibentry{25e}
summarizes my reaction to events at Glacier View. My 38-page paper \bibentry{25f}
was distributed as an official Glacier View
document. My 14-page \bibentry{25g}
summarizes responses to 125 questions.
The poll was sent to a list of all Bible scholars in North America (teaching
and non-teaching) provided by the GC Department of Education, and to several
overseas. This report includes, also, a list of responses to a 1958 poll I
sent to 27 teachers of Hebrew in North American SDA colleges, and a few
others proficient in Hebrew, all personal friends of mine. 
} in Colorado, to serve as the
Sanctuary Review Committee. They were specifically instructed not to
evaluate Ford's beliefs with respect to Daniel 8:14, the sanctuary, and the
investigative judgment by the Bible itself, but as set forth in the
statement of Twenty-seven Fundamental Beliefs, which the church had already
determined to be normative. Several weeks later the Australasian Division
withdrew his ministerial credentials.

Procedures at Glacier View consisted of a reaffirmation of the traditional
Adventist interpretation of Daniel 8:14. But Ford was given no opportunity
to present the reasons for his ``apotelesmatic'' interpretation of it, which
provided for the traditional Adventist interpretation being one of several
fulfillments of the prophecy, but not the fulfillment. Again---as
always---the church neglected to examine the reasons for dissent from the
traditional interpretation of Daniel 8:14 and merely reaffirmed it in
stentorian tones. As a matter of fact, the consensus report voted at the
close of the week-long conference tacitly agreed with Ford on six major
points of exegesis. Later, some forty Bible scholars signed a document known
as the Atlanta Affirmation, remonstrating with Neal Wilson for the way the
church had treated Ford at, and after, Glacier View.

In his involuntary ``retirement'' Ford has continued to proclaim the gospel,
in a ministry he called ``Good News Unlimited.'' Unlike Canright, Ballenger,
and others before him who had embarked on vendettas against the church, Ford
has remained a dedicated Seventh-day Adventist at heart and retained his
church membership.\sidenote{Ford is still a member of the Pacific Union Church}

Ford, now retired in his native Queensland, Australia, is the lone survivor
of numerous traumatic encounters with the traditional interpretation of
Daniel 8:14. We could wish that such encounters with the sanctuary doctrine
were a thing of the past. But a new generation of victims is repeating their
traumatic experiences all over again. If the past is any index to the future
they will be repeated indefinitely unless and until the church faces up to
the facts objectively and deals realistically and responsibly with them
in harmony with the sola Scriptura principle. 

It is said that more than 150 ordained ministers, mostly in Australia,
forfeited their ministerial credentials in the aftermath of the Ford affair.
Hundreds of lay persons, mostly in the United States, left the church and
formed effervescent ``fellowships'' as a result.

Dale Ratzlaff was pastor of the Watsonville church in the Central California
Conference and a Bible teacher at nearby Monterey Bay Academy when, in 1981,
he was abruptly fired by the Conference for expressing a conviction shared
by a majority of the forty or so Bible scholars at Glacier View, that
administration had misjudged and mistreated Desmond Ford the year before.
The elders of the Watsonville church invited Dr.\ Fred Veltman of Pacific
Union College and me to meet with the church the following Sabbath, in which
we endeavored to pour oil on the troubled waters.

Ratzlaff left the Adventist church and wandered about (both geographically
and ideologically) for a few years following which he embarked on what he
calls Life Assurance Ministries, first in Sedona and now in Glendale,
Arizona, with the objective of warning Adventists and others against the
church. First came a 350-page polemic against the Sabbath, and in 2001 the
384-page Cultic Doctrine of Seventh-day Adventists, which he describes as
``an appeal to SDA leadership.'' His target in Cultic Doctrine is the
traditional Adventist Interpretation of Daniel 8:14, the sanctuary doctrine,
and the investigative judgment. In 1999 he began publishing Proclamation, a
bi-monthly journal dedicated to warning Adventists and others against
Adventism. Here in the West, Dale's crusade is having at least a measure of
success. He is also publisher of Dr.\ Jerry Gladson's 383-page A Theologian's
Journey From Seventh-day Adventism to Mainstream Christianity (copyright
2001).\sidenote{\bibentry{27a} focuses on the traditional Adventist doctrine
of the sanctuary. \bibentry{27b} is an account of obscurantist leadership
persecution as a result of the traditional sanctuary doctrine.}

Dr.\ Jerry Gladson had the very considerable misfortune to serve on the
faculty of Southern Adventist College (now University). Had he been teaching
at any of the other eight Adventist colleges or universities in North
America he would probably still be an Adventist minister and teacher.
Southern operates as an agency of Southern Bible belt obscurantism.
Furthermore it was (and still is) to an appreciable extent, dependent on the
largesse of committed ultra-fundamentalists, who insist that the college
operate on ultra-fundamentalist principles. Again the target was the
traditional sanctuary doctrine and the charge what Gladson thought about it,
not anything he had taught in his classes.

Then dean of the Adventist Theological Seminary Dr.\ Gerhard F.\ Hasel, a
former student and teacher at Southern and the ruthless personification of
Adventist obscurantism, played an active role in the lynching of Dr.\
Gladson, a role in which Hasel had already distinguished himself at the
Seminary. The head of the religion department at Southern, responsible for
the ultimate coup de grace, was as closed-minded and ruthless as Torquemada, 
a role in which he had already distinguished himself as director of the
Biblical Research Institute of the General Conference. What chance did Dr.\
Gladson have for a fair evaluation and adjudication of the charges against
him? Finally, the chairman of the college board distinguished himself as
either a committed obscurantist or a willing instrument of the far Adventist
right.

Jerry Gladson was not fired nor were his ministerial credentials withdrawn.
He remained an ordained minister until they expired and were not renewed.
Instead, a witch-hunting climate was created in which departure proved to be
the lesser of two evils. There was no formal hearing. No one tried to
understand his reasons for thinking as he did, or cared. The Pharisees were
in control, and that was that. An anomalous situation indeed!27

Janet Brown became a Seventh-day Adventist in 1985. As a lay person she was
an avid Bible student, and as such ``began to notice more and more problems
and inconsistencies between SDA teachings and the Bible.'' For a time she
ignored these ``cracks in the armor of Adventism,'' but as ``the evidence
really began to pile up'' she felt that she could no longer ``remain honest''
with herself and continue as a Seventh-day Adventist. To her, the
investigative judgment resembles Roman Catholic purgatory inasmuch as it
keeps people in suspense as to their standing before God and ``makes no sense
biblically.'' In 1995 she left the Adventist church and operates a website
devoted to opposing it.\sidenote{Janet Brown gives her email as Janet.E.Brown@intel.com}

Don W. Silver of Ashland Kentucky is another lay person who left Adventism
recently, primarily because of the sanctuary doctrine, which he vehemently
opposes. Evidently well-educated, he speaks with fervor and pin-point logic.
His wife, like him well-educated, teaches at nearby Marshall University and
remains a faithful Adventist and a leader in the local Adventist church.
Their two grown daughters have followed their father into
agnosticism.\sidenote{Mrs.\ Donald W.\ Silver (Christine M.\ Silver) is the daughter
of Dr.\ and Mrs.\ Robert H.\ Brown.}

Other contemporary illustrations of opposition to the sanctuary doctrine and
resulting apostasy might, of course, be cited. I know personally of other
employees of the church who have been fired for the same reason, of lay
people who have left the church, and of families that have been broken up as
a result. The sanctuary problem is still with us, late and soon, and is
touching the lives of sincere Seventh-day Adventists. 

\chapter{Non-Adventist Reaction to the Sanctuary Doctrine}
\label{ch:non-sda}

\newthought{It was the sanctuary doctrine} based on Daniel 8:14 that made us Seventh-day
Adventists and that remains, today, the keystone of our distinctive belief
system and our mission to the world. Of it, Ellen White wrote: ``The
Scripture which above all others had been both the foundation and central 
pillar of our faith was the declaration, `Unto two thousand and three
hundred days; then shall the sanctuary be cleansed'\,''\cite{8} and ``The correct
understanding of the ministration in the heavenly sanctuary is the
foundation of our faith.'' ``Not one pin is to be removed from that which the
Lord has established. The enemy will bring in false theories, such as the
doctrine that there is no sanctuary. This is one of the points on which
there will be a departing from the faith.''\cite{31}

When, in the mid-1950's, Walter Martin and Donald Grey Barnhouse explored
Adventist teachings in depth with persons appointed by the General
Conference, they concluded that, with two exceptions, we are in harmony with
the gospel:
\begin{enumerate}
    \item our sanctuary doctrine, and 
    \item the role we popularly
ascribe to Ellen White as an infallible interpreter of Scripture, in
contradiction of her own explicit statements to the contrary.
\end{enumerate}
The former,
they concluded, violates the Reformation principle sola
Scriptura.\sidenote{My 28-page paper \bibentry{32a} is a detailed review of the
eighteen Martin-Barnhouse interviews with General Conference personnel in 1955
and 1956.  My 10-page \bibentry{32b} recounts a number of humorous moments during
the Martin-Barnhouse interviews.} Of it, Barnhouse wrote: 
\begin{quote}
The [sanctuary] doctrine is, to me, the most colossal,
psychological, face-saving phenomenon in religious history. \ldots We
personally do not believe that there is even a suspicion of a verse in
Scripture to sustain such a peculiar position, and we further believe that
any effort to establish it is stale, flat, and unprofitable. \ldots [It is]
unimportant and almost naïve.\cite{33} 
\end{quote}

\newpage
Such is the usual reaction of non-Adventist Bible scholars and other
biblically literate non-Adventists to our sanctuary doctrine.\sidenote{My
16-page \bibentry{34} quotes from, and summarizes, comment on the contemporary
(1956) Evangelical Christian press regarding the Martin-Barnhouse interviews.
This document was prepared as the request of the editorial committee preparing
\bibentry{34a} for publication.}

\chapter{My Personal Encounter With the Sanctuary Doctrine}
\label{ch:personal}

\newthought{I first encountered problems} with the traditional interpretation of Daniel
8:14, professionally, in the spring of 1955 during the process of editing
comment on the Book of Daniel for volume 4 of the SDA Bible Commentary. As
a work intended to meet the most exacting scholarly standards, we intended
our comment to reflect the meaning obviously intended by the Bible writers.
As an Adventist commentary it must also reflect, as accurately as possible,
what Adventists believe and teach. But in Daniel 8 and 9 we found it
hopelessly impossible to comply with both of these requirements.\sidenote{My
article \bibentry{35} classifies and summarizes
some five thousand Old Testament passages relating to God's dealings with
Israel under the covenant relationship, including the Old Testament
perspective of salvation history, which culminated in the coming of Messiah
and the establishment of His eternal reign of righteousness at or soon after
the close of Old Testament times. These five thousand passages were
accumulated during the course of teaching the class Old Testament Prophets
for several years at Pacific Union College during the 1940s and 1950s. The
parenthetical sentence on page 38, ``This rule does not apply to those
portions of the book of Daniel that the prophet was bidden to shut up and
seal, or to other passages whose application Inspiration may have limited
exclusively to our own time,'' was added by F. D. Nichol during the editorial
process. He personally agreed with everything in the article and made no
alterations in it, but feared for the adverse reception of the Commentary
except for this caveat.}

In 1958 the Review and Herald Publishing Association needed new printing
plates for the classic book Bible Readings, and it was decided to revise it
where necessary to agree with the Commentary. Coming again to the Book of
Daniel I determined to try once more to find a way to be absolutely faithful
to both Daniel and the traditional Adventist interpretation of 8:14, but
again found it impossible. I then formulated six questions regarding the
Hebrew text of the passage and its context, which I submitted to every
college teacher versed in Hebrew and every head of the religion department 
in all of our North American colleges---all personal friends of mine.
Without exception they replied that there is no linguistic or contextual
basis for the traditional Adventist interpretation of Daniel 8:14.\sidenote{See
note 26}

When the results of this questionnaire were called to the attention of the
General Conference president, he and the Officers appointed the super-secret
Committee on Problems in the Book of Daniel, of which I was a member.
Meeting intermittently for five years (1961-1966), we considered 48 papers
relative to Daniel 8 and 9, and in the spring of 1966 adjourned sine die,
unable to reach a consensus.\sidenote{My set of the committee papers considered is in the GC
Archives.}

The Commentary experience with Daniel already mentioned led me into an
unhurried, in-depth, spare-time, comprehensive study of Daniel 7 to 12
that continued without interruption for seventeen years (1955-1972), in
quest of a conclusive solution to the sanctuary problem. My objective was to
be fully prepared with definitive, objective, biblical information the next
time the question should arise during the course of my ministry for the
church.

Among other things I memorized, in Hebrew, all relevant portions of Daniel 8
to 12 for instant recall and comparison (60 verses), conducted exhaustive
word studies\sidenote{My study of 150 important words in the Aramaic and Hebrew
portions of Daniel fills 108 typewritten pages.}
of more than 150 relevant Hebrew words Daniel uses,
throughout the Old Testament, studied the Hebrew grammar and syntax in
detail, made a minute analysis of contextual data,\sidenote{My correlation 
of the prophecies of Daniel 7, 8, 9, and
11-12 fills 14 typewritten pages.} compared ancient Greek
and Latin translations of Daniel,\sidenote{For my own convenience, I wrote
out (in parallel columns)
key passages of the prophecies of Daniel in Hebrew, Greek (both the LXX and
Theodotion), the KJV, and RSV.} investigated relevant apocryphal and New
Testament passages,\sidenote{Especially the first four chapters of 1 Maccabees, where I
found twenty-four points of specific identity between Daniel's little horn
and the career of Antiochus IV Epiphanes. I concluded, however, that Christ
assigned the fulfillment of Daniel's prophecies to New Testament times, and
that the New Testament writers nearly forty times anticipate Jesus' promised
return within their generation. Chapters 4 ``The Old Testament Perspective of
Salvation History'' and 12 ``The New Testament Perspective of Salvation
History'' in my unpublished book manuscript \bibentry{41} sets
all of this forth in detail. See Note 131.} traced Jewish and Christian
interpretation of Daniel
from ancient to modern times,\sidenote{Chapter 13 of \bibentry{41} traces Jewish
interpretation in some detail from ancient to modern times. For this I relied 
primarily on \bibentry{42a}, \bibentry{42b}, and \bibentry{42c}.}
and made an exhaustive study of the
formation, development, and subsequent Adventist experience with the
traditional sanctuary doctrine.\sidenote{Chapter 14 of \bibentry{41} traces
the development of the traditional Adventist 
interpretation of Daniel 8:14 in considerable detail.}
Eventually I incorporated the results of
this investigation into an 1100 page manuscript which I later reduced to 725
pages but decided not release for publication until an appropriate time.

The above considerations conclusively demonstrate that our traditional
interpretation of Daniel 8:14, the sanctuary, and the investigative judgment
as set forth in Article 23 of Fundamental Beliefs does not accurately
reflect the teaching of Scripture with respect to the ministry of Christ on
our behalf since His return to heaven.\sidenote{Chapter 17 of \bibentry{41}
explores its comment on Christ's
ministry in the heavenly sanctuary in considerable detail.}Accordingly, 
it is appropriate
\begin{enumerate}
	\item to note wherein Article 23 is thus defective,\sidenote{see~\ref{ch:flaws}}
	\item to revise the article so
as to reflect Bible teaching on this aspect of His ministry accurately, and
	\item to suggest a process designed to protect the church from this and
similar traumatic experiences in the future.\sidenote{see~\ref{ch:remedy}}
\end{enumerate}

Some of the concepts associated with the investigative judgment are, indeed,
biblical, but the Bible itself nowhere associates them with an investigative
judgment, for which there is no sola Scriptura basis whatever.\sidenote{See 44}

Upon ascending to heaven Jesus assured His disciples ``I am with you always,
to the end of the age'' (Matthew 18:20). The Book of Hebrews is our primary
source of information about His ministry in heaven on their (and our) behalf
since that time, I suggest that the following composite summary of His
ministry as presented in Hebrews provides an appropriate basis for a revised
article 23 of Fundamental Beliefs, should such a statement eventually be
desired. The author of Hebrews presents Christ's ministry in heaven, on our
behalf, by analogy with the role of the high priest in the ancient sanctuary
ritual: 
\newpage
\begin{quote}
On the cross Jesus offered Himself as a single sacrifice for all time that
atoned for the sins of those who draw near to God through Him.
\sidenote[][-20pt]{See Heb. 7:27 in note 49. Heb. 10:11-12 Day after day every 
priest stands and performs his religious duties; again and again he offers the 
same sacrifices, which can never take away sins. But when this priest had 
offered for all time one sacrifice for sins, he sat down at the right hand of God,}
That one
sacrifice qualified Him to serve as our great High Priest in heaven,
perpetually.\sidenote{Heb. 2:17-18 For this reason he had to be made like 
them, fully human in every way, in order that he might become a merciful and 
faithful high priest in service to God, and that he might make atonement for 
the sins of the people. Because he himself suffered when he was tempted, he 
is able to help those who are being tempted. See Heb. 4:14-15 in note 51.
Heb. 6:19-20 We have this hope
as an anchor for the soul, firm and secure. It enters the inner sanctuary behind 
the curtain, where our forerunner, Jesus, has entered on our behalf. He has 
become a high priest forever, in the order of Melchizedek. Heb. 7:24-28 but 
because Jesus lives forever, he has a permanent priesthood. Therefore he is able 
to save completely those who come to God through him, because he always lives 
to intercede for them. Such a high priest truly meets our need--one who is holy, 
blameless, pure, set apart from sinners, exalted above the heavens. Unlike the 
other high priests, he does not need to offer sacrifices day after day, first 
for his own sins, and then for the sins of the people. He sacrificed for their 
sins once for all when he offered himself. For the law appoints as high priests 
men in all their weakness; but the oath, which came after the law, appointed 
the Son, who has been made perfect forever.}
Having made that sacrifice, Christ 
entered the Most Holy
Place--``heaven itself''--to appear in the presence of God on our behalf.
\sidenote{See Heb. 7:25 in note 49.  Heb. 9:12
He did not enter by means of the blood of goats and calves; but he entered the 
Most Holy Place once for all by his own blood, thus obtaining eternal redemption.
Heb. 9:24 For Christ did not enter a sanctuary made with human hands that was 
only a copy of the true one; he entered heaven itself, now to appear for us in 
God's presence.}
He invites us to come boldly to Him, by faith, to find mercy and grace to help
us in our time of need.\sidenote{See Heb 2:17-18 in note 49. Heb. 4:14-16 Therefore, since 
we have a great high priest who has ascended into heaven, Jesus the Son of 
God, let us hold firmly to the faith we profess. For we do not have a high 
priest who is unable to empathize with our weaknesses, but we have one who has 
been tempted in every way, just as we are--yet he did not sin. Let us then 
approach God’s throne of grace with confidence, so that we may receive mercy 
and find grace to help us in our time of need.} He will soon appear, a second 
time, ``to bring salvation to those who are waiting for him.''\sidenote{Heb. 9:28
so Christ was sacrificed once to take away the sins of many; and he will appear 
a second time, not to bear sin, but to bring salvation to those who are waiting 
for him. Heb. 10:37 For, ``In just a little while, he who is coming will come
and will not delay.''}
\end{quote}

