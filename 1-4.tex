\chapter{Continuing Casualties of the Sanctuary Doctrine}
\label{ch:cont}

\newthought{Like an airplane} unexpectedly entering a region of clear air turbulence, in
1945 Dr.\ Desmond Ford began to encounter exegetical problems in the
traditional Adventist interpretation of Daniel 8:14, the sanctuary, and the
investigative judgment. He set out to put all of the disparate pieces
together in a coherent pattern that would resolve the problems, that would
be faithful to reliable principles of exegesis, and that left him a
dedicated Seventh-day Adventist with complete confidence in the integrity of
the church as an authentic witness to the everlasting gospel.

Over the next ten or fifteen years Ford discovered that some of his
contemporaries and others before him had wrestled with the same problems. In
his definitive 991-page Glacier View document, Daniel 8:14, the Day of
Atonement, and the Investigative Judgment, he names twelve Adventist Leaders
with whom he had discussed the problems, in person or by correspondence. He
devoted his master's and one of his doctoral dissertations to the subject.
His published commentaries on the Books of Daniel and the Revelation total
more than two thousand pages. He has probably devoted more scholarly study
to the subject and written more extensively on it than any other person in
history.

During his long tenure as head of the theology department at Avondale
College in Australia he trained half or so of the ministers in Australia. In
the classroom and by his personal example he inspired thousands of young
people for Christ. He was always in demand as a speaker, and thousands
testify to a clearer understanding and appreciation of the gospel as a
result of his witness to it. His theme ever was---and still is---salvation
by faith in Jesus Christ.

Ford never discussed the controversial aspects of the sanctuary doctrine in
public---until October 27, 1979, as an exchange professor at Pacific Union
College, when several members of the faculty invited him to discuss his
views on the sanctuary question in an open meeting one Sabbath afternoon.
Thirty-four years of silence on the subject surely reflect commendable
pastoral and scholarly restraint. The PUC presentation ``was positive on the 
providential role of Adventists and Ellen White.'' However, three retired
ministers present detected what they perceived to be heresy and reported
their version of his remarks to the chairman of the college board. 

In view of the fact that Ford was still an employee of Avondale College in
Australia and due to return to Avondale at the close of the 1979--1980 school
year, the chairman logically referred the matter to the General Conference.
In August 1980 115 leading administrators and Bible scholars from around the
world (at an administrator's estimated cost of a quarter of a million
dollars) were summoned to Glacier View\sidenote{For a summary of highlights
of the 991-page \bibentry{24}, 
see my
18-page paper, \bibentry{25b}. For a very
detailed account of proceedings at the Glacier View meeting of the Sanctuary
Review Committee, August 10--15, 1980, see my report \bibentry{25c}.
This article is based on my complete shorthand notes of every speech and all
proceedings at the morning Study Group 2, of which I was a member, and the
afternoon and evening plenary sessions. My unpublished 20-page paper \bibentry{25d}
explains what happened at Glacier View and why it
happened as it did. My 21-page unpublished paper \bibentry{25e}
summarizes my reaction to events at Glacier View. My 38-page paper \bibentry{25f}
was distributed as an official Glacier View
document. My 14-page \bibentry{25g}
summarizes responses to 125 questions.
The poll was sent to a list of all Bible scholars in North America (teaching
and non-teaching) provided by the GC Department of Education, and to several
overseas. This report includes, also, a list of responses to a 1958 poll I
sent to 27 teachers of Hebrew in North American SDA colleges, and a few
others proficient in Hebrew, all personal friends of mine. 
} in Colorado, to serve as the
Sanctuary Review Committee. They were specifically instructed not to
evaluate Ford's beliefs with respect to Daniel 8:14, the sanctuary, and the
investigative judgment by the Bible itself, but as set forth in the
statement of Twenty-seven Fundamental Beliefs, which the church had already
determined to be normative. Several weeks later the Australasian Division
withdrew his ministerial credentials.

Procedures at Glacier View consisted of a reaffirmation of the traditional
Adventist interpretation of Daniel 8:14. But Ford was given no opportunity
to present the reasons for his ``apotelesmatic'' interpretation of it, which
provided for the traditional Adventist interpretation being one of several
fulfillments of the prophecy, but not the fulfillment. Again---as
always---the church neglected to examine the reasons for dissent from the
traditional interpretation of Daniel 8:14 and merely reaffirmed it in
stentorian tones. As a matter of fact, the consensus report voted at the
close of the week-long conference tacitly agreed with Ford on six major
points of exegesis. Later, some forty Bible scholars signed a document known
as the Atlanta Affirmation, remonstrating with Neal Wilson for the way the
church had treated Ford at, and after, Glacier View.

In his involuntary ``retirement'' Ford has continued to proclaim the gospel,
in a ministry he called ``Good News Unlimited.'' Unlike Canright, Ballenger,
and others before him who had embarked on vendettas against the church, Ford
has remained a dedicated Seventh-day Adventist at heart and retained his
church membership.\sidenote{Ford is still a member of the Pacific Union Church}

Ford, now retired in his native Queensland, Australia, is the lone survivor
of numerous traumatic encounters with the traditional interpretation of
Daniel 8:14. We could wish that such encounters with the sanctuary doctrine
were a thing of the past. But a new generation of victims is repeating their
traumatic experiences all over again. If the past is any index to the future
they will be repeated indefinitely unless and until the church faces up to
the facts objectively and deals realistically and responsibly with them
in harmony with the sola Scriptura principle. 

It is said that more than 150 ordained ministers, mostly in Australia,
forfeited their ministerial credentials in the aftermath of the Ford affair.
Hundreds of lay persons, mostly in the United States, left the church and
formed effervescent ``fellowships'' as a result.

Dale Ratzlaff was pastor of the Watsonville church in the Central California
Conference and a Bible teacher at nearby Monterey Bay Academy when, in 1981,
he was abruptly fired by the Conference for expressing a conviction shared
by a majority of the forty or so Bible scholars at Glacier View, that
administration had misjudged and mistreated Desmond Ford the year before.
The elders of the Watsonville church invited Dr.\ Fred Veltman of Pacific
Union College and me to meet with the church the following Sabbath, in which
we endeavored to pour oil on the troubled waters.

Ratzlaff left the Adventist church and wandered about (both geographically
and ideologically) for a few years following which he embarked on what he
calls Life Assurance Ministries, first in Sedona and now in Glendale,
Arizona, with the objective of warning Adventists and others against the
church. First came a 350-page polemic against the Sabbath, and in 2001 the
384-page Cultic Doctrine of Seventh-day Adventists, which he describes as
``an appeal to SDA leadership.'' His target in Cultic Doctrine is the
traditional Adventist Interpretation of Daniel 8:14, the sanctuary doctrine,
and the investigative judgment. In 1999 he began publishing Proclamation, a
bi-monthly journal dedicated to warning Adventists and others against
Adventism. Here in the West, Dale's crusade is having at least a measure of
success. He is also publisher of Dr.\ Jerry Gladson's 383-page A Theologian's
Journey From Seventh-day Adventism to Mainstream Christianity (copyright
2001).\sidenote{\bibentry{27a} focuses on the traditional Adventist doctrine
of the sanctuary. \bibentry{27b} is an account of obscurantist leadership
persecution as a result of the traditional sanctuary doctrine.}

Dr.\ Jerry Gladson had the very considerable misfortune to serve on the
faculty of Southern Adventist College (now University). Had he been teaching
at any of the other eight Adventist colleges or universities in North
America he would probably still be an Adventist minister and teacher.
Southern operates as an agency of Southern Bible belt obscurantism.
Furthermore it was (and still is) to an appreciable extent, dependent on the
largesse of committed ultra-fundamentalists, who insist that the college
operate on ultra-fundamentalist principles. Again the target was the
traditional sanctuary doctrine and the charge what Gladson thought about it,
not anything he had taught in his classes.

Then dean of the Adventist Theological Seminary Dr.\ Gerhard F.\ Hasel, a
former student and teacher at Southern and the ruthless personification of
Adventist obscurantism, played an active role in the lynching of Dr.\
Gladson, a role in which Hasel had already distinguished himself at the
Seminary. The head of the religion department at Southern, responsible for
the ultimate coup de grace, was as closed-minded and ruthless as Torquemada, 
a role in which he had already distinguished himself as director of the
Biblical Research Institute of the General Conference. What chance did Dr.\
Gladson have for a fair evaluation and adjudication of the charges against
him? Finally, the chairman of the college board distinguished himself as
either a committed obscurantist or a willing instrument of the far Adventist
right.

Jerry Gladson was not fired nor were his ministerial credentials withdrawn.
He remained an ordained minister until they expired and were not renewed.
Instead, a witch-hunting climate was created in which departure proved to be
the lesser of two evils. There was no formal hearing. No one tried to
understand his reasons for thinking as he did, or cared. The Pharisees were
in control, and that was that. An anomalous situation indeed!\sidenote{See note 27}

Janet Brown became a Seventh-day Adventist in 1985. As a lay person she was
an avid Bible student, and as such ``began to notice more and more problems
and inconsistencies between SDA teachings and the Bible.'' For a time she
ignored these ``cracks in the armor of Adventism,'' but as ``the evidence
really began to pile up'' she felt that she could no longer ``remain honest''
with herself and continue as a Seventh-day Adventist. To her, the
investigative judgment resembles Roman Catholic purgatory inasmuch as it
keeps people in suspense as to their standing before God and ``makes no sense
biblically.'' In 1995 she left the Adventist church and operates a website
devoted to opposing it.\sidenote{Janet Brown gives her email as Janet.E.Brown@intel.com}

Don W. Silver of Ashland Kentucky is another lay person who left Adventism
recently, primarily because of the sanctuary doctrine, which he vehemently
opposes. Evidently well-educated, he speaks with fervor and pin-point logic.
His wife, like him well-educated, teaches at nearby Marshall University and
remains a faithful Adventist and a leader in the local Adventist church.
Their two grown daughters have followed their father into
agnosticism.\sidenote{Mrs.\ Donald W.\ Silver (Christine M.\ Silver) is the daughter
of Dr.\ and Mrs.\ Robert H.\ Brown.}

Other contemporary illustrations of opposition to the sanctuary doctrine and
resulting apostasy might, of course, be cited. I know personally of other
employees of the church who have been fired for the same reason, of lay
people who have left the church, and of families that have been broken up as
a result. The sanctuary problem is still with us, late and soon, and is
touching the lives of sincere Seventh-day Adventists. 
