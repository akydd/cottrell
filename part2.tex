\part{A Sola Scriptura Examination of the Doctrine}
\label{p:sola}

\chapter{``Rightly Explaining the Word of Truth''}
\label{ch:rightly}
\marginnote{2 Timothy 2:15 Biblical hermeneutics has been the focus of
my study for more than fifty years, the chapter ``Principles of Biblical
Interpretation'' in \bibentry{53a}, being one of my first published papers
(1953) in this area. Among my many papers on the subject have been
\bibentry{53b} (37 pages), \bibentry{53c} (43 pages), \bibentry{53d} (18 pages),
and many others.}

\newthought{The almost infinitely diverse} and often contradictory ideas attributed to
the Bible, and thus its relevance for our time, suggest the importance of
identifying principles on the basis of which we can have confidence in the
validity of our conclusions with respect to the perspectives of life and
reality its divine Author and the inspired writers intended their words to
convey.

We read and study the Bible with the objective of learning who we are, how
and why we came to be here, how we should relate to life and make the most
of its opportunities, where we are going, and how best to get there. This
constitutes what we may call our ``world view'', our concept of what life on
planet Earth is all about.

Our quest for this information is something like a literal journey from
where we may be now to where we would like to be, but have never been over
the road before. In planning such a journey we must first know where we are,
where we want to be at journey's end, and the best way to get there. Our
planning must take into consideration the facts of geography and travel as
they really are, not as we might like or imagine them to be. In other words
we must be objective with respect to reality, to the facts of geography and
travel as they really are. To be subjective in our planning---to think of
them as we might imagine or like them to be---could eventually prove to be
disastrous. It is the same with reading and studying the Bible: Objectivity
is essential. Being subjective in our study and thinking inevitably imposes 
our personal, unenlightened, opinions upon the Bible and leaves us blind and
deaf to what God is trying to say to us through it. As a result, we assume
that our personal opinions constitute the voice of God!

In the Bible even a child or a semi-literate person can find the way of
salvation and follow it all the way to the pearly gates, and find welcome
there. But for in-depth study of some portions of it those not at home with
ancient Hebrew and Greek should make use of relevant reference material
prepared by reliable persons who are conversant with those languages.
Certain factors are essential for everyone conducting in-depth study of the
Bible. The following is a brief resume of factors essential to such a study.

Objectivity is the mental quality that aspires to evaluate ideas and draws
conclusions in terms of their intrinsic reality, rather than in terms of a
person's untested, subjective presuppositions. Objectivity is essential for
ascertaining the intended import of the Bible.

Untested, subjective presuppositions regarding the nature and teachings of
the Bible almost inevitably lead to wrong conclusions. Everyone, consciously
or unconsciously, comes to the Bible with a set of presuppositions about it
which control evaluation of the data considered and thus the conclusions
drawn from it. Accordingly, the importance of presuppositions is crucial in
determining the validity of one's conclusions. Presuppositions should ever
remain open to revision as clearer, objective evidence may require. The
objective is to eliminate every subjective factor from the reasoning process
in order to bring it into harmony with objective reality.

Is it possible to test the presupposition that the Bible is, as it claims to
be, the unique revelation of God's infinite will and purpose for the human
race? Yes. The objective evidence for this consists of 
\begin{enumerate}
    \item the Bible's
accurate evaluation of the natural human ethical-moral-spiritual state, 
    \item its perfect remedy for the imperfections of that natural state,
    \item the
demonstration that that remedy has transformed the psyche of countless
millions of human beings for two thousand years, and
    \item that if Bible
principles were universally accepted and practiced they would automatically
eliminate all war, all crime, and all selfish manipulation of other human
beings---and thus transform this world into a little heaven on earth!
\end{enumerate}
Given
the opportunity, the human experience confirms these conclusions beyond the
possibility of either doubt or error. This authenticates Bible principles as
being of more than human origin, and so validates the above presupposition
as being objective and trustworthy.

The Old Testament was written between twenty-four and thirty-seven centuries
ago, mostly in ancient Hebrew and in a world more than a little different
and strange to us. The New Testament was written in Greek some nineteen
centuries ago. The Old Testament records the history of the Hebrews as the 
covenant people and chosen instrument of the divine purpose for them and for
the human race in ancient times, instruction designed to qualify them to be
living representatives of, and witnesses for, the true God, and their
individual and corporate response to this instruction.\sidenote{See note 35} The Hebrew language
had a limited vocabulary that reflected their primitive culture and world
view, a form of writing that consisted of consonants only, and grammar and
syntax different from ours today.

The Bible was thus historically conditioned,\cite{55} that is, adapted and
specifically addressed to, the needs, comprehension, and covenant role of
its recipients at the time it was written, and to their circumstances and
perception of the divine purpose, yet Its fundamental principles and
instruction are of universal value and applicability. It was written in
their language and in thought forms with which they were familiar, and
reflects the salvation history perspective of their time. That record,
however, ``was written for our instruction'' also. Accordingly, we need to
historically condition our minds to their time, circumstances, and
perspective of salvation history in order to fully understand and appreciate
its message for our time. In-depth study and appreciation of the Bible
require that the historical circumstances in which a passage was written
must be taken into consideration.

The salvation history perspective of the Old Testament envisioned ancient
Israel as God's covenant people and chosen instrument of the divine purpose
to restore humanity to harmony with the divine purpose for this world.\sidenote{See note 35} God
revealed all of this to them in order that they might cooperate
intelligently with His infinite purpose for the human race. That revelation,
imparted over the centuries of antiquity, provided ancient Israel with
instruction that would qualify them individually and collectively as a
nation to fully represent the supreme value and desirability of cooperating
with His eternal purpose. It envisioned the climax of earth's history and
the complete restoration of divine sovereignty over all the earth at the
close of Old Testament times. The New Testament assumes the validity of this
Old Testament perspective of salvation history as reaching a climax in the
life, ministry, crucifixion, resurrection, and promise of Jesus to return
soon---at the close of New Testament times.\sidenote{See chapter 12 of \bibentry{41}.
Nearly forty times the New Testament writers anticipate the return of Christ
within their generation.  See note 131.}

This Bible perspective of salvation history was implicit in Scripture and in
the minds of people of that time. It must also be in our minds as we read
Scripture. Accordingly, the salvation history perspective of the time a
passage was written must be taken into consideration in order to ascertain
its intended, true meaning.

The original text of Scripture, in the languages in which it was written, is
the ultimate, supreme authority for what it says.\sidenote{I relied on
\bibentry{58a}, \bibentry{58b}, and \bibentry{58c}, eleven volumes of which are
available in English.}Good modern translations
such as the New Revised Standard Version (NRSV)\sidenote{Except as otherwise 
noted I used the Revised Standard
Version of the Bible, but often referred to other translations.}, the New International 
Version (NIV), and the Good News Bible (Today's English Version, TEV) are as
accurate and reliable translations as any available today. The King James
Version (KJV), with its superb, stately literary style has had a profound
influence on the English language and endeared itself to readers for nearly
four centuries, but sometimes it does not accurately reflect the original
text.\sidenote{Two problems limit the value of the King James' Version for
serious study: (1) it was based on late manuscripts that had accumulated a
considerable number of scribal errors, and (2) several hundred English words
convey a different meaning today than they did in 1611. \bibentry{60}
explains several hundred English
words in the KJV that are either obsolete or archaic today.}

This was because the KJV was based on late manuscripts that had accumulated
numerous scribal errors and editorial changes over several centuries since
the original autographs. Since an ancient manuscript known as the Sinaiticus
was discovered in 1844, thousands of ancient manuscripts centuries closer to
the originals have been found that provide us, today, with much more
accurate information as to how the original autographs actually read.\sidenote{Footnotes
in \bibentry{58a} list numerous helpful variant readings in the ancient 
versions and translations of the Hebrew Bible.}
Also, the biblical languages are better understood than they were in 1611,
when the KJV became available, and the history and culture of antiquity are
better understood. Word studies---the way in which Hebrew and Greek words
occur in the Bible and their meaning as defined by context, in each
instance---are thus essential to determine their meaning.

The literary context of a passage is essential to an accurate determination
of its meaning. This includes its immediate context, in particular, but also
its extended context in the entire document of which it forms a part.
Ancient Hebrew, in which most of the Old Testament was written,\sidenote{My 
knowledge of Aramaic is limited.} had
already become a dead language to the extent that when Ezra read from ``the
book of the law of Moses'' (the Torah, or Pentateuch) in public about 450
B.C., it needed interpretation in order for Jews, even of his time, to
understand it.\sidenote{Neh. 8:7-8 The Levites—Jeshua, Bani, Sherebiah, 
Jamin, Akkub, Shabbethai, Hodiah, Maaseiah, Kelita, Azariah, Jozabad, Hanan 
and Pelaiah—instructed the people in the Law while the people were standing 
there. They read from the Book of the Law of God, making it clear and giving 
the meaning so that the people understood what was being read.}

Several characteristics of ancient Hebrew were responsible for this:
\begin{enumerate}
    \item For
one thing, it had a very limited vocabulary, one in which many words were
used to express a wide variety of meanings. (For instance, the KJV
translates ten common Hebrew words by an average of eighty-four English
expressions each, and one of them by 164 English words and
expressions!\cite{64}). 
    \item Ancient Hebrew writing consisted of consonants only,
and the reader had to supply whatever vowels he thought were intended, and
in some instances might supply a set of vowels different from those the
writer intended.\sidenote{In the ancient Hebrew of Genesis 1:1 the word for ``created''
was written br' (consonants only). The Masoretes supplied vowels to make it
read bara', ``created.'' With equal reason they might as well have supplied
vowels to make it read bore', which would have verse 1 read ``When God began
to create \ldots '', thus making verse 1 a dependent clause, with verse 2 the
main statement} The vowels that now appear in Hebrew Bibles were added to
its consonants by the Masoretes, Jewish scholars, many centuries after
ancient Hebrew had become a dead language, according to what they thought to
be the intended meaning.
\end{enumerate}
For this reason it is futile to correlate two
passages of scripture on the basis of the same English word located in a
concordance---as William Miller did in developing the sanctuary doctrine! 

The analogy of Scripture---the use of one Bible passage to clarify
another---must be used with caution.\sidenote{See Section 7, on the analogy 
of Scripture. The heavenly
sanctuary of the Book of Hebrews is not a valid counterpart for the
sanctuary of Daniel 8:14 because because verses 9 to 13 identify it as the
sanctuary located in the ``beautiful'' land (tsebi), Judea. Chapter 11:16, 41
confirms this identification, and in 11:45 tsebi the `beautiful' holy
mountain in Jerusalem where the temple was located. Furthermore, context
(8:11-13) specifically identifies the reason the sanctuary needs ``cleansing''
or restoration because of its trampling by the little horn (cf. 11:31). } 
The context of both passages must 
first be taken into account in order to determine whether or not they may be
used together.

In summary, in-depth study of the Bible requires consideration of one's
presuppositions, the historical circumstances to which a passage was
addressed and to which it was intended to apply, its salvation history
perspective, its sense as determined by the original language, its literary
context, and cautious use of other Bible passages of Scripture to amplify
it.

Seventh-day Adventists today affirm the sola Scriptura principle of the
Reformation in principle, but sometimes unwittingly compromise it in
practice, notably in affirming the traditional interpretation of Daniel
8:14.

Seventh-day Adventism emerged as a discrete entity within the Christian
community on October 23, 1844\sidenote{The name ``Seventh-day Adventists'' 
was chosen in 1860, and
the General Conference was organized in 1863.} as the result of a particular understanding
of Daniel 8:14 and the great disappointment that attended their
disillusionment the preceding day. That understanding, which was
subsequently modified in some details and became the traditional Adventist
interpretation, has, since then, been considered the keystone of Adventism's
self-identity, understanding of the Bible, theology, and sense of mission.
\sidenote{See \nameref{ch:egw}.  I have explored Adventism's sense of mission 
is \bibentry{68}.}
In Jeremiah 18:7-10 the prophet summarizes the nature and purpose of
predictive prophecy as follows: 
\begin{quote}
At one moment I may declare concerning a nation or a
kingdom, that I will pluck up and break down and destroy it, but if that
nation concerning which I have spoken, turns from its evil I will change my
mind about the disaster that I intended to bring on it. And at another
moment I may declare concerning a nation or kingdom that I will build and
plant it, but if it does evil in my sight, not listening to my voice, then I
will change my mind about the good that I intended to do to it. 
\end{quote}

Accordingly, predictive prophecy is always conditional on the response of
the people to whom it is addressed. Its function is not to demonstrate
divine foreknowledge nor does it necessarily predetermine the course of
events, for if it did it would thereby deprive people of the power of
choice. Its intended purpose is to enable them to make wise choices in the
present by indicating the ultimate result of either a right or a wrong
choice. For this reason Bible prophecy, even apocalyptic prophecy, is always
conditional, and its time element is always flexible, in order to provide
for the free exercise of human choice.\sidenote{In Moses' farewell address 
to Israel prior to their entrance
into the promised land (Deuteronomy 28) he set forth the good things that
would happen to them if they obeyed God's instructions (verses 1-14), and
the misfortunes if they disobeyed (verses 15-68). The argument that Daniel 8
and 9 are ``apocalyptic'' (and thus supposedly immune to the conditionalism
principle) ignores the fact that, contextually, they apply specifically to
the Hebrew people and therefore are subject to the conditions specified in
Jeremiah 18:7-10.}It is a preview of what can be, not
what necessarily will be.

Accordingly, the seventy weeks-of-years of Daniel 9:24-27 provided the
Hebrew exiles in Babylon with a preview of what the future held for them, 
subject to their cooperation.\sidenote{See previous note.}

\section{Three Methods of Bible Study}

The traditional Adventist interpretation of Daniel 8:14 was formulated on
the basis of what is commonly known as the prooftext method of biblical
study and interpretation, which construes Bible passages in terms of what a
modern reader thinks to be their import. This method
\begin{enumerate}
    \item is highly
subjective,
    \item understands the Bible from the modern reader's cultural,
historical, and salvation history perspectives,
    \item accepts the Bible in
translation as authoritative,
    \item makes the reader's personal and
group-think presuppositions normative for evaluating data and for
    \item
drawing conclusions.
\end{enumerate}
This method does not require special training or
experience, and is followed by a majority of untutored Bible readers. Since
the beginning most Adventists have followed this method, but no reputable
Bible scholar follows it today. 

When Daniel 8:14 is studied by the historical method, serious flaws in the
traditional interpretation become apparent because the historical method
\begin{enumerate}
    \item
aspires to be as objective as possible,
    \item endeavors to understand the
Bible as the various writers intended what they wrote to be understood and
as their original reading audience would have understood it from their
cultural, historical, and salvation history perspective,
    \item considers
words, literary forms, and statements according to their meaning in the
original language as normative,
    \item endeavors to evaluate data objectively,
and
    \item bases its conclusions on the weight of evidence.
\end{enumerate}
This method
requires either special training in biblical languages and the history and
milieu of antiquity, or reliance on source material prepared by persons with
such training. Since about 1940 most Adventist Bible scholars have followed
this method.

Since about 1970 a hybrid of these two methods known as the
historical-grammatical method\cite{71} has attained limited popularity among
Seventh-day Adventist Bible scholars and lay people, and major support among
church administrators. Why? It consists of historical method procedures
under the control of prooftext presuppositions and principles, which enable
it to provide apparent scholarly support for traditional conclusions. It is
highly subjective, aspires to dominate and eventually control all official
Adventist study of the Bible, and has more or less controlled General
Conference doctrinal policy for the past thirty years.

Let us emulate the sincerity and diligence of our spiritual forefathers in
their study of God's Word. We have no valid reason to criticize them because
of the flaws we find in their understanding of the Bible.\sidenote{Reading 
one of William Miller's books, I found his
uninterrupted misuse of commonly accepted principles of exegesis a deeply
troubling experience.} Let us remember
that they did the best they knew how as they studied the Bible by the 
prooftext method, the generally accepted method of that time.\sidenote{For
characteristics of the prooftext method, see Section 7} They did not
have access to the more accurate ancient Bible manuscripts that we do today,
nor to our knowledge of ancient Hebrew and Greek or the history of ancient
times. In taking note of flaws in the traditional interpretation of Daniel
8:14 we can be grateful for their dedication, build on their labors, and be
faithful in our time as they were in theirs, to the best it is our privilege
to know.\sidenote{For a list of changes the church has already made in the
Sanctuary doctine see~\bibentry{74a} or~\bibentry{74b}.}

\chapter{``Rightly Explaining'' Daniel 8:14}
\label{ch:explain}

\newthought{The first imperative} for comprehending the prophecies of Daniel in the sense
Inspiration intended is an objective frame of mind divested of every
personal, subjective, modern presupposition with respect to their import.

The second imperative is to identify the circumstances set forth in Daniel 1
to 6 and 9:1--23, which provide the historical background within which
Inspiration set its five prophetic passages and from which it intended
Daniel and his intended readers to understand them. Accordingly, in order to
understand those passages as Inspiration intended them to be understood we
must do so with that historical perspective in our minds, and from the same
perspective of salvation history as Daniel and his intended readers did. Any
interpretation that ignores or controverts that historical perspective and /
or the salvation history perspective of their time is automatically suspect

The first six chapters of the Book of Daniel recount the exile of Daniel and
his compatriots to Babylon ``in the third year of the reign of Jehoiakim of
Judah,'' which is dated to 606/5 B.C., and their experiences during the
seventy years of exile foretold by Jeremiah in chapter 29:1-14. According to
Daniel 9:1, in ``the first year of Darius'' (which is dated to 537/6 B.C. by
Jewish inclusive reckoning), Daniel had been in exile for exactly seventy
years. But as yet there was no visible evidence that release from exile was
imminent. Accordingly, Daniel prayed the importunate prayer for release from
exile and for restoration recorded in chapter 9:4-19. 
and imposes an alien, uninspired interpretation on those prophecies. 

While Daniel was still praying the angel Gabriel reappeared75 and said, ``I
have now come out to give you wisdom and understanding. At the beginning of
your supplications a word went out [obviously in heaven], and I have come to
declare it, for you are greatly beloved. So consider the word and understand
the vision.'' Gabriel thereupon repeats that ``word'' verbatim (verse 24), as
he had promised, and proceeds to explain it in verses 25 to 27. 

It is of crucial importance to note that Gabriel explicitly identifies the
``word'' that ``went out to restore and build Jerusalem'' at the commencement of
the seventy weeks of years as ``the word'' that ``went out''--in heaven--while
Daniel was praying.76 That ``word''77 was obviously one that only God Himself 
(and not an earthly monarch) could possibly have issued! On the authority of
no less a person than the angel Gabriel, the ``seventy weeks'' of years thus
began in 537 B.C., not eighty years later in 457 B.C.!

Gabriel's explanation of that ``word'' in verses 25--27 very briefly sketched
the future of God's covenant people during the seventy weeks of years, and
its climax in the ruthless oppression of ``the prince who is to come'' during
the seventieth of the seventy ``weeks,'' which he had already foretold in
chapter 8:9--13 and explained in verses 19 to 25.78 

As already noted, Daniel 9:23-25 begins the seventy weeks of years at the
time the ``word'' was issued in heaven, in 537 B.C. In the same way,
contextual identification of the ``he'' of verse 27 identifies events of
history that mark their close in the seventieth of the seventy ``weeks.'' It
is universally accepted that the immediate antecedent of a personal pronoun
identifies the person to whom it refers unless the context unambiguously
specifies otherwise. Accordingly, verse 26 identifies the immediate
antecedent of the pronoun ``he'' in verse 27, who ``make[s] a strong covenant
with many'' for the seventieth of the seventy ``weeks'' and ``make[s] sacrifice
and offering cease'' during the last half of the ``week,'' as the evil ``prince
who is to come''--not the ``anointed prince'' of verses 25--26! 

Chapter 11:23 confirms the fact that his alias, the last king of the north,
does, indeed make such a covenant with people in ``alliance'' with him. Also,
his fate set forth in verse 27, ``the decreed end is poured out on the
desolator,'' is equivalent to the horn-king of chapter 8:25 being ``broken,
and not by human hands,'' and to the last king of the north in chapter 11 who
``come[s] to his end, with no one to help him.''79 

Chapter 9:24--27 thus provides an exact but much more complete explanation of
chapter 8:13-14's question and answer about events between Daniel's time and
``the appointed time of the end'' ``many days from now'' when ``the vision of the
evenings and the mornings'' was to meet its fulfillment.80 Isn't that exactly
what Gabriel said the audition of 9:24-27 was supposed to do?81 

Such is Daniel's perspective of salvation history. In order to understand
chapters 8 and 9 as heaven intended them to be understood, we must imagine
ourselves in Daniel's historical circumstances and view them from his
perspective of salvation history in order to form an accurate understanding
of what was revealed to him.

\section{Daniel's Perspective of Salvation History}

Daniel's perspective of salvation history was a composite of the visions of
chapters 2 and 7, each with its explanation, and chapter 8 with
its three-fold explanation in chapters 8, 9, and 11-12. It consisted of a 
series of universal kingdoms82 followed by a period of disintegration and
fragmentation,83 which Gabriel told Daniel would be a ``troubled time''
(9:25)84.

At the ``appointed time of the end \ldots many days from now''--after sixty-nine
of the ``seventy weeks of years''85--there would be an unprecedented ``time of
anguish'' for God's people in which they would be ``trampled,'' their power
shattered,86 their land and city devastated,87 their loyalty and
faithfulness to God tested,88 their covenant with Him and its prescribed
system of worship abolished,89 and an idolatrous system of worship
enforced.90 As a result of this attempt to obliterate the knowledge and
worship of the true God, many Jews would apostatize and enter into a
"covenant" with their oppressor.91

The duration of this time of anguish for God's people is given variously as
\begin{enumerate}
    \item ``a time, two times, and half a time'' = three and a half years,92 as
    \item the last half of the seventieth of the seventy ``weeks'' = also three and a
half years,93 and as 
    \item  the time during which 2300 evening and morning
sacrifices would normally have been offered = 1150 literal days = three
years, two months, and 10 days94 within the three and a half years of
``anguish.''95 
\end{enumerate}

At the close of this time of anguish the Ancient of days would sit in
judgment and ``the decreed end'' would be ``poured out upon the desolator,'' who
would thus ``come to his end with no one to help him'' and be ``broken'' but
``not by human hands.``96 Simultaneously, the sanctuary would ``be restored to
its rightful state,`` the Ancient of Days would vindicate His faithful people
and award them an ``everlasting kingdom,`` Michael would arise to deliver
them, the righteous dead would be raised to life eternal, the ``wise,''
including Daniel, would enter upon their eternal reward and shine like the
brightness of the firmament for ever and ever.97

The prophecies of Daniel locate this time of anguish 
\begin{enumerate}
    \item during the ``time,
two times, and half a time'' of Daniel 7:25, 
    \item at or near ``the end'' of the
``rule'' of the four horn Greek era of chapter 8:8, 21-23, 
    \item during the last
half of the seventieth of the seventy weeks of chapter 9:24-27, and 
    \item during the reign of the last king of the north of chapter 11:20-45. 
\end{enumerate}

Obviously Daniel's perspective of salvation history was vastly different
from ours--by more than two thousand years! But by the sure word of his
angel mentor that was the perspective from which he and the angel Gabriel
then viewed the future. It is the identical format set forth in the Old
Testament.35 To ignore or deny it is a major violation of the sola Scriptura
principle, and to say that neither Daniel nor Gabriel knew what they were
talking about! It is an important part of in-depth study of the Bible to 
read it from its own historical and salvation history perspectives, in order
to understand and appreciate its message for us in our time!


