\chapter{The Daniel and Revelation Committee}
\label{ch:committee}

Eventually realizing that Glacier View had not settled the sanctuary issue,
the General Conference appointed the Daniel and Revelation Committee (DRC)
and assigned it the task of compiling what was intended to be definitive 
proof of the traditional interpretation of Daniel 8:14, the sanctuary, and
the investigative judgment. The committee functioned during the 1980s under
the auspices of the General Conference Biblical Research Institute (BRI) and
published its report in seven volumes under the title Daniel and Revelation
Committee Series (DRCS).

The five volumes of the DRCS series devoted to Daniel defend what is now
considered the official response of the church to all questions regarding
the sanctuary doctrine. Unwittingly, however, DRCS presents Adventist
scholarship under the control of obscurantism. It does not address any of
the contextual anomalies to which Chapter 8 above, \nameref{ch:explain},
calls attention!

One would have expected so important a committee as DRC to be composed, at
least primarily, of a cross-section of the trained, experienced, known, and
trusted Bible scholars of the church. It was not! They were intentionally
excluded! The composition, or membership, of the committee bears the
unmistakable imprint of Gerhard Hasel as the only one who could have
selected its members. Why so? At the time, he was dean of the Theological
Seminary, at the height of his career, and approximately half of DRC's
eighteen members had been Seminary students during his fifteen years or so
as a member of the Seminary faculty. They were otherwise unknown to either
the General Conference or the incumbent Bible scholars in the colleges of
North America. And they all shared Hasel's hermeneutical perspective, as did
all but three other members of the committee!

As reflected in the DRCS report the conclusions to which the committee came
with respect to the sanctuary doctrine were thus determined before the
committee ever met!

As set forth in the preface to volume 1 of the series, its interpretation is
based on the historicist principle of prophetic interpretation, with respect
to which it acknowledges that ``Seventh-day Adventists stand virtually alone
as exponents'' today. Historicism interprets the predictive prophecy of the
Bible as providing an uninterrupted continuum of fulfillment from Bible
times to the present. In so doing it rejects the Bible's own, inherent,
perspective of salvation history, which explicitly anticipates the climax of
earth's history, Christ's promise to return, and the establishment of God's
eternal, righteous dominion over all the earth at the close of Bible
times.\sidenote{See note 35} The DRCS reaffirmation of historicism is the crux of the issue to
which this paper is addressed. It is the ultimate, ``scholarly,''
demonstration of the perennial obscurantism that has characterized
Adventism's perennial reaffirmation of the sanctuary doctrine for more than
a century.

It is not the objective of this paper to review the five DRCS Daniel volumes
in detail, but rather to evaluate the credibility of its historicist
interpretation in terms of faithfulness to the sola Scriptura principle and
to generally recognized principles of exegesis, particularly the crucial
importance of context. Most of its 1600 pages are devoted to scholarly
analyses of the text of Daniel that only a trained Bible scholar would be
able to evaluate. Others would probably depend on their personal
presuppositions with respect to the sanctuary doctrine in accepting or
rejecting the conclusions to which the respective authors draw from the
evidence they present.

1519 of the 1600 pages consist of articles by 18 authors. One author
contributed 418 pages (28\%), another 176 pages (12\%), and a third 111 pages
(9\%), for a total of 705 pages. The other 15 authors contributed an average
of 54 pages each, five of them as little as 12 pages or less.

The disorganized way in which DRCS deals with the sanctuary doctrine
reflects the disorganized way in which its parent ``committee'' (DRC) must
have operated. A committee is expected to integrate the contributions of its
members into a consensus that represents the committee as a committee. A
Bible translation conducted by a group of translators working together is
considered to be far more accurate and reliable than one by a single
individual, however qualified that individual may be. The consensus of the
group tends to eliminate individual idiosyncrasies, however ``scholarly'' they
may be. DRCS offers no such consensus or synthesis.

The eighteen DRCS authors are to be commended for their knowledge of ancient
and recent literature relevant to the prophecies of Daniel, for their
expertise in ancient Hebrew and cognate languages, and for their obviously
diligent labors encapsulating all of this for modern readers. On the other
hand, their labors were flawed because of their obviously overriding
subjective use of this information in defense of an interpretation of the
prophecies of Daniel that, as a matter of fact, contradicts what Daniel
intended what he wrote to convey, as determined by context.158

Almost without exception the DRCS authors tacitly assume the validity of the
historicist principle as their fundamental presupposition and then,
reasoning in a circle, offer what they write as proof of that
presupposition! At four major points they assume the accuracy of the KJV
translation where it misrepresents the Hebrew text. They ignore the
historical context within which Daniel locates his visions and to which he
applies them, and his explicit, composite, salvation history perspective. In
at least seven major instances they ignore or contradict Daniel's explicit
statements in the context. And in the year of our Lord 2002 BRI, with the
full approval of the GC, affirms DRCS as final and conclusive proof of the
traditional understanding of Daniel 8:14, the sanctuary, and the 
investigative judgment! Reductio ad absurdum and the ultimate exercise in
obscurantism posing as the highest level of scholarship Adventists have to
offer!\sidenote{See Chapter 8, \nameref{ch:explain}}

In another noteworthy anomaly, the several chapters dealing with the
supposed analogies between the sanctuary of Daniel 8:14 and the sanctuaries
of the books of Leviticus and Hebrews is based on the supposition that its
sanctuary is the heavenly sanctuary, whereas, as noted in section 8 above,
context explicitly identifies it as the sanctuary, or temple, in Jerusalem.
These two analogies are valid only if the context in Daniel permits them. It
does not, period! Thus the several chapters devoted to the sanctuary in
Leviticus and Hebrews are irrelevant to the exegesis of Daniel 8:14! 

Dr.\ William Shea's protracted and convoluted chiastic literary analysis of
significant passages of Daniel throughout volume one of the DRCS and
elsewhere, sometimes in explicit contradiction of context, may be impressive
to the uninitiated but wearisome beyond measure and otherwise
counterproductive. DRCS would have been vastly improved without his 418
pages of comment! Much of Dr.\ Gerhard Hasel's 176 pages consists of
detailed analyses of non-Adventist interpretations of Daniel that are of no
value or relevance to any Seventh-day Adventist studying the book of Daniel.
Accordingly, some 40\% of DRCS's 1519 pages of comment is really of little or
no practical value with respect to clarifying the Adventist understanding of
its prophetic pericopes. In many respects DRCS is a mute witness to the
uncoordinated and irrelevant way in which DRC evidently functioned, yet BRI
informs us that it has settled, once for all, every question about the
traditional interpretation of Daniel 8:14, the sanctuary, and the
investigative judgment!

Currently in progress is another General Conference project which seems
destined to solidify the Pierson-Hyde-Hasel objective of transforming the
Seventh-day Adventist Church from a community dedicated and open to the
continued guidance of the Holy Spirit into an ever more accurate and
complete ``knowledge of our Lord and Savior Jesus Christ,''\sidenote{2 Peter 3:18 
But grow in the grace and knowledge of our Lord and Savior Jesus Christ. To him 
be the glory both now and to the day of eternity. Amen.} into the
closed, obscurantist, fundamentalist church that they envisioned--the
International Board for Ministerial Training and Endorsement with its
sub-boards in the various divisions. This project is already proving to be
divisive, and has the possibility of repeating the fate that overtook the
Lutheran Church--Missouri Synod in December 1976--schism.163
