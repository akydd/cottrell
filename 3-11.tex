\chapter{Obscurantism and the Sanctuary Doctrine}
\label{ch:obscurantism}

Webster defines obscurantism as ``depreciation of or positive opposition to
enlightenment or the spread of knowledge, esp.\ a policy \ldots of deliberately
making something obscure or withholding knowledge from the general public.''
Here, the word ``obscurantism'' is used in the specific sense of making
presumably authoritative decisions and/or statements with respect to the
sanctuary doctrine on the basis of untested, preconceived opinions and/or
without first weighing all of the available evidence on the basis of sound,
recognized principles of exegesis and basing conclusions exclusively on the
weight of all the evidence.

Obscurantism has characterized the official response of the church to every
question raised with respect to the traditional interpretation of Daniel
8:14, the sanctuary doctrine, and the investigative judgment. In at least
most instances this obscurantism has been inadvertent and not intentional,
but its effect has been the same as if it had been intentional. It is high
time for the church to be done with the traditional clichés with which it
has heretofore responded to questions regarding the sanctuary doctrine. It
is time to face up to and to deal fairly and objectively with all of the
evidence.

\subsection{A Window of Hope and Opportunity at Mid-Century}

Elder R.\ R.\ Figuhr's twelve years as president of the General Conference at
mid-century (1954-1966) provided the church with an era of wise leadership
and openness in which administrators and trained Bible scholars worked
together harmoniously and effectively in resolving biblical and doctrinal
questions. Over the preceding fifteen years the church had developed a
community of trained, responsible Bible scholars whose professional
expertise Elder Figuhr respected and trusted, and who, in turn, respected
and appreciated his wise leadership. An open, happy, and rewarding working
relationship developed between them that was good for the church.

Another important aspect of that mid-century era of good will and
cooperation was the spirit of consensus and harmony among the Bible scholars
of the church in which the sometimes bitter doctrinal factionalism\sidenote{\bibentry{68},
pages 34-54} of the
earlier decades of the century had disappeared. For this two factors were
responsible, the first being the Bible Research Fellowship, pioneer
professional organization of Bible Scholars, and second, the SDA Bible
Commentary.

At their 1940 meeting in Takoma Park the North American college Bible
teachers authorized the formation of a professional organization in which
they could work together on matters of exegesis and doctrine, share the
results of their study with one another, and benefit from one another's
constructive criticism.\sidenote{See~\bibentry{134a}, p. 32 and \bibentry{134b},
p. 74 (in the GC archives)} This organization became a reality three years
later--1943--in the Bible Research Fellowship (BRF),\sidenote{See previous 
note} of which Dr.\ L.\ L.\
Caviness was chairperson and I secretary throughout its brief lifetime of
approximately ten years. We were teaching together in the religion
department at Pacific Union College.

Eventually, BRF membership rose to 250 and, with one exception, included all
college level Bible Teachers around the world. Many others, including
seventeen General Conference persons, were dues-paying members. During
those
ten years more than 90 formal papers were considered and shared with
members.\sidenote{My complete file of BRF papers is in the Heritage Room of 
the James White Memorial Library at Andrews University. (During the first year
or two of our monthly Sabbath afternoon meetings at PUC some presentations were
oral only, without formal papers.)} At the Bible teachers' 1950 meeting at 
Pacific Union College,
responses to a questionnaire found complete agreement with respect to every
major, divisive exegetical and doctrinal issue over the preceding fifty
years!\sidenote{See previous note for the 1950 meeting} At that 1950 meeting 
BRF made a report of its operations, a formal
vote of appreciation for BRF was taken, and all joined in singing the
Doxology.

In 1951, on behalf of BRF, I proposed to the General Conference that it
establish a permanent committee to replace 
BRF.\sidenote{\bibentry{137}.
I sent this proposal to Le Roy Froom, founder of
the Ministerial Association and a personal friend of mine for 28 years; R.
Allen Anderson, incumbent director of the Ministerial Association; and W. E.
Read.}
The 1952 Autumn (now
Annual) Council accepted my proposal and established the Biblical Research
Committee (BRC) of the General Conference. Thereupon Dr.\ Caviness, present 
as a delegate, formally handed over BRF operations to BRC. Simultaneously
transferring from Pacific Union College to the Review and Herald Publishing
Association to edit the Bible Commentary, I was appointed a charter member
of BRC. After several years, for a still higher level of continuity and
effective service to the church, I proposed that the committee become an
institute.\sidenote{\bibentry{138a}. Appended to it was \bibentry{138b}, in
which I reviewed  events of the years
1940 to 1966. The appendix was intended to provide him with information 
about what had happened in Adventist Bible scholarship during his protracted
absence.} This was voted in 1975, whereupon BRC became the Biblical
Research Institute (BRI), which it remains today (2002).

The second unifying factor was production of the seven-volume SDA Bible
Commentary (1952-1957),\sidenote{\bibentry{139}. The Commentary did not identify
authors because of numerous editorial changes made in some contributions. My
Spectrum article lists all the contributors.} in which a team of approximately 
fifty writers and editors participated.\sidenote{See previous note} Prior to 
publication each volume was read and
criticized by ten church leaders around the world, who were paid for their
criticisms.\sidenote{See page 10 of any volume of the commentary} Some 
critical sections were read and criticized by 125 such
readers. All criticisms were carefully evaluated, and where considered
appropriate, accepted.

But during the late 1960's that brief mid-century era of openness, good
will, progress, and cooperation between administrators and Bible scholars
began imperceptibly to erode into the closed-minded, polarized,
obscurantist, and theological witch-hunting that continues to the present
time (2002). In order to understand this subtle change in the Adventist
climate over the past thirty years, let us note first, the three architects
of obscurantism primarily responsible for it. All three were southern Bible
belt fundamentalists. We will also note several specific evidences of
obscurantism.

\subsection{Architects of Obscurantism}

The role of this part of Section 11 on obscurantism in the church over the
past 33 years is to explain how the present climate of obscurantism
surreptitiously invaded and captured the church. Only a person who served
the church through the preceding era of openness and mutual respect between
administrators and Bible scholars at the General Conference level is in a
position to appreciate the profound change that revolutionized Adventist
theology, Biblical hermeneutics, and approach to doctrine during the decade
of obscurantism (1969-1980).

The three principal architects of obscurantism introduced briefly below were
all obviously sincere, dedicated individuals who conscientiously believed
that their ultimate objective, or ``end,'' justified whatever means they might
employ to achieve that objective. For instance, they were never willing to
enter into open, responsible dialog with those who did not share their
perspective, but two of the three always, consistently put daggers in the
backs of those whom they suspected of not sharing their point of view. In
personal conversation the president of the General Conference admitted this
to me.

On the contrary, it was my privilege to converse personally with each of the
``architects of obscurantism'' named below, by which I came to understand
their objectives and methods first hand. Realizing, eventually, that the
last two of the three were simply implementing Elder Pierson's policy and
objectives, I spent many hours at various times in conversation with him,
the last being two or three hours on the chartered Pan-American flight
returning from the General Conference Session in Vienna, in 1975.

These conversations were always positive, ``friend of the court'' in tone in
which I dealt with principles and never mentioned anyone's name. In one of
those conversations Elder Pierson cryptically told me that one of the other
two ``architects'' was disseminating (among GC personnel) inaccurate
accusatory comments with respect to loyal Adventist scholars whom he
considered theological renegades. In our correspondence following Elder
Pierson's retirement in 1979 we both expressed appreciation for each other's
friendship. In his last letter a short time before his death he wrote:
``Through the years that we served together in Washington I always considered
you as a friend. Though there may have been areas of differing opinions I
had a warm feeling for you personally.'' In my last letter to him I expressed
the same sentiment.

Robert H. Pierson was a gracious person, a dedicated Adventist, a gentleman
in every way, but also a person with clear objectives and resolute
determination to achieve them. A major objective of his administration as
president of the General Conference was to replace the administrator / Bible
scholar partnership that had developed during Elder Figuhr's administration
with strict administrative control of the theological and doctrinal
processes of the church.

During his thirteen years as president of the General Conference (1966-1979)
Elder Pierson completely reversed the policy of his predecessor, R. R.
Figuhr, with respect to biblical studies, doctrine, and cooperation with its
community of Bible scholars. His very sincere but resolute aim was to
restore the situation that had prevailed when he graduated from Southern
Junior College in 1933 and left North America three years later for
distinguished overseas service in India, the Caribbean, and South Africa,
where he served with distinction until he was elected GC president thirty
years later. For all practical purposes, in 1936 church administrators had
been in exclusive control of theology and doctrine for the church. At that
time there were no trained Adventist Bible scholars. Anyone who attended an
``outside'' university for training in such subjects as biblical languages,
archeology, ancient history, and chronology was automatically considered
persona non grata by every Adventist college board.\sidenote{Among the first 
Adventist ``Bible teachers,'' as Bible scholars were
then called, to attend ``outside'' universities were: R.\ E.\ Loasby, E.\ C.\
Banks, S.\ H.\ Horn, W.\ G.\ C.\ Murdoch, E.\ R.\ Thiele, L.\ H.\ Wood, and A.\ G.\
Maxwell. They tended to avoid classes in theology as such, but focused on
such subjects as biblical languages, the history of antiquity, archeology,
and chronology}

Accordingly, Pierson distrusted the entire Adventist community of Bible
scholars and set out to exclude them from meaningful participation in the 
Biblical and doctrinal deliberations of the church. In private conversation
and in GC committees he repeatedly stated it to be his policy that
administrators alone--and not in counsel with Bible scholars--should
decide exegetical questions for the church. His first step toward
implementing this policy took place at the Spring Meeting of the GC in 1969,
which eliminated the Bible scholars of the church, en masse, from the
Biblical Research Committee\sidenote{General Council Spring Meeting minutes 
for April 4, 1969.}--a policy that was never implemented, however,
due to vigorous protests from the Theological Seminary faculty. Undaunted,
however, later that year he achieved his objective by adding numerous
administrators and other non-scholars to BRC, and appointing a vice
president of the GC to supervise the Biblical Research Committee (now
Institute) and the GC office of biblical studies (BRI).\sidenote{In the autumn 
of 1968 R.\ H.\ Pierson invited W.\ J.\ Hackett to serve
as a GC vice president. They had become acquainted on the 1968 Geoscience
field trip of that summer. Elder Hackett confided in me that one of his
principal objectives was to ``clean up'' the religion faculties at Loma Linda
and Andrews universities.}

Also in the spring of 1969, Pierson invited a teacher at his alma mater,
Southern Adventist College (now University), to chair BRC--Gordon M.
Hyde---whose training was in communication--and who shared Pierson's
Southern Bible belt fundamentalist theological perspective. Hyde protested
that he was not trained in theology, but Pierson explained that he was to
function as an administrator and not as a Bible scholar.\sidenote{A personal 
friend of mine, a colleague then on the religion faculty
at Southern Adventist College, shared this information with me.} With this
understanding Hyde accepted the invitation, and when, during his first years
at the GC he was expected to reply to a theological question, he parried the
question with the explanation that he was not a theologian.

Upon occasion Hyde could be devious and underhandedly maneuver to achieve
his objectives. For instance, at the week-long GC-appointed Charistmatic
Committee at Camp Cumby-Gay in Georgia, Hyde announced that every speaker
was to confine his remarks to thirty minutes. But he gave Hasel two full
hours for his presentation. Upon another occasion he invited Hasel to a
sensitive subcommittee hearing to which the Bible Research Committee had
explicitly not appointed him, and provided him with copies of papers to be
presented to that subcommittee which were to be shared with the appointed
members of the committee only. Members of the subcommittee objected to this
faux pas on Hyde's part, and as a result the subcommittee never met.\cite{145}

When, toward the close of my forty-seven years of service to the church Hyde
repeatedly refused requests for a face-to-face reconciliation, I wrote him a
nine-page letter ``looking for reconciliation'' in which I mentioned the
problems that had arisen between us and made a final appeal for an
opportunity to restore the friendly relationship we had enjoyed when he
first came to the GC. But he never replied and was intransigent against ever
meeting.

Hyde's major project designed to promote Hasel as leading theologian of the
church was the series of three North American Bible Conferences, the first
of which convened at Southern Adventist College, the second at Andrews
University, and the third at Pacific Union College. He assigned Hasel the 
theme topic, biblical hermeneutics, and featured him on every panel
discussion. The senior members of the Theological Seminary faculty were
bypassed altogether or assigned relatively minor 
roles.\sidenote{For example, W.\ G.\ C.\ Murdoch, S.\ H.\ Horn, E.\ E.\ Heppenstall.}

Hyde's attempt to have Hasel appointed dean of the Theological Seminary in
the spring of 1974 (prior to the conferences) was aborted by the senior
members of the faculty because of Hasel's interference with established
Seminary procedures, his collusion with Gordon Hyde and the GC to control
Seminary policy, and what the senior members of the faculty referred to as
his ``intolerable
dogmatism.''\sidenote{In personal conversation with W. G. C. Murdoch, 
Siegfried H. Horn, and E. E. Heppenstall, long-time personal friends of mine.} Hasel did, however, become dean in 1980, but
was demoted seven years later for plagiarism and his attempt to separate the
Seminary from Andrews University.

Without expertise in biblical studies and theology himself, Hyde selected
Gerhart F. Hasel, a former colleague at Southern Adventist College who had
transferred to the Seminary in 1967 and whose ultra-conservative perspective
he shared, as his mentor and personal adviser in biblical-theological
matters. Hyde's objective was to elevate Hasel to be the leading Adventist
theologian and dean of the Theological Seminary at Andrews University, where
he would be in a position to indoctrinate the next generation of Adventist
Bible scholars and pastors with his obscurantist hermeneutical perspective.

During his tenure as dean, Hasel made several teachers more experienced than
he feel unwelcome at the Seminary and, in effect, froze them out--Drs.
Sakai Kubo, Ivan Blazen, Fritz Guy, and Larry Geraty. All four were
immediately invited to serve at other Adventist institutions of higher
education, three of them as college or university presidents. Hasel
forthwith appointed Seminary students he had trained, and who accepted his
biblical hermeneutic, to replace them. He and Gordon Hyde subsequently
forced two other religion faculty members--Drs. Lorenzo Grant and Edwin
Zachrison--to leave Southern Adventist College at approximately the same
time as Jerry Gladson, and the president of the college resigned in protest.
Hasel never approached his targets directly, in compliance with Matthew
18:15, but stuck verbal daggers in their back by denouncing them to
administrators (who accepted his word without verifying it).

Over the decade 1969 to 1979 this triumvirate--Pierson, Hyde, and
Hasel--conspired effectively together to gain control of Adventist Biblical
studies, theology, and doctrine in harmony with their fundamentalist,
obscurantist perspective.\sidenote{See Chapter 9, \nameref{ch:flaws}} Hasel's 
role was to control Adventist biblical
studies and theology. Hyde's role was to devise procedures by which to
implement Hasel's hermeneutical and theological perspective, Pierson's role
was to protect Hasel and Hyde whatever they might attempt to do. I have set
forth a documented record of thirty-one specific incidents in this
conspiracy designed to implement Pierson's policy, in my forty-page paper
Architects of Crisis: A Decade of Obscurantism (1969-1979).

This explains the origin of the obscurantist climate in the church over the
past thirty years and its unwillingness to deal objectively with the
numerous exegetical anomalies in the traditional Adventist interpretation of
Daniel 8:14 with its sanctuary and investigative judgment.

\subsection{Aftermath of the Decade of Obscurantism}

By the close of the decade of obscurantism (1969-1979) the goal of its three
architects was firmly in place. Elder Pierson, ailing, retired a year early.
Replaced as director of BRI, Gordon Hyde transferred to Southern Adventist
College to be dean of the School of Religion. Gerhard Hasel became dean of
the Theological Seminary for seven years (1980-1987), after which the
General Conference demoted him, primarily because of his attempt to separate
it from Andrews University.\sidenote{In conversation with a long-time personal 
friend of mine, then in
the inner circle of ATS leadership. He confided to me the fact that ATS was
organized specifically as a result of Hasel's loss of influence when demoted
from deanship of the Theological Seminary}
That unanticipated event precipitated the
founding of the Adventist Theological Society (ATS) the following year
(1988), which was specifically designed to perpetuate the objectives of the
decade of obscurantism in view of Hasel's loss of influence as Seminary
dean.\sidenote{My paper \bibentry{71} evaluates the history and objectives of 
ATS. The section on ATS hermeneutics is based on personal interviews and official 
ATS publications.}

In view of the fact that Gordon Hyde was then dean of the school of religion
at Southern College (SC; now University) and Gerhard Hasel dean
of the Theological Seminary at Andrews University, between 1980 and 1987,
that both had been teachers at SC prior to 1969, and that Robert Pierson
was a graduate (1933) of Southern when it was a junior college, it was no
accident that the Adventist Theological Society (ATS) was founded at SC in
1988 by representatives of both institutions and that SC became its first
headquarters until it later moved to Andrews University. Thus ATS has a
solid basis in Adventist Southern Bible belt fundamentalism, which
determines its hermeneutical and theological orientation.\sidenote{See previous note}

Developments at the General Conference (GC) level since the decade of
obscurantism (1969-1979) are likewise intimately related to these facts.
Among these developments have been the following:
\begin{enumerate}
	\item obscurantism in control at Glacier View,\sidenote{See note 25} 
	\item obscurantism in relating to Walter Rea,\sidenote{\bibentry{68}, pages 49-50}
	\item obscurantism at Consultations I and II,\sidenote{For Consultation I 
	see \bibentry{153a}; for Consultation II see \bibentry{153b}}
	\item obscurantism in the Daniel and Revelation Committee and its 5-volume 
	report,\sidenote{\bibentry{154}, volumes 1-5} 
	\item obscurantism in the Methods of Bible Study 
	report,\sidenote{\bibentry{155} notes the fact that after the committee 
	released its report BRI inserted a preamble reiterating ATS hermeneutical 
	principles. As a result some members of the committee told me that they had 
	refused to sign their names in approval of the document. ATS requires 
	members to affirm acceptance of it.}
	\item obscurantism at the GC Biblical
	Research Institute, and thus in control of GC doctrinal policy,\sidenote{Personal 
	correspondence with both the former and the new (2002) BRI
	directors and the president of the GC makes evident that they are firmly
	committed to ATS hermeneutical policy.}
	\item obscurantism in the way several dissenting faculty members at the 
	Seminary and SAC have been treated,\sidenote{For instance, Drs. Fritz Guy, 
	Larry Geraty, Sakai Kubo, and Ivan Blazen (at the Theological Seminary); 
	and Drs. Lorenzo Grant, Edwin Zachrison, and Jerry Gladson (at Southern 
	Adventist University).}
	\item obscurantism motivating the present GC
	(IBMTE) and NAD committees formulating a low-tolerance-level policy with
	respect to dissent from official doctrinal policy.
\end{enumerate}
The triumvirate has proved to be eminently successful!

\subsection{The Nature and Raison d'Etre of Doctrinal Obscurantism}

Obscurantism is unwillingness to examine either purported or demonstrated
facts objectively, and to encourage or coerce others into accepting
subjective presuppositions. The classic illustration of obscurantism was
president of the Flat Earth Society Simon Voliva's journey around the world
in 1929, when upon his return he explained to society members that his trip
had proved conclusively that planet earth is flat--by going in a circle on
its flat surface!

Obscurantism is the result of a subjective state of mind in which one's
unproved presuppositions take precedence over the weight of objective
evidence to the contrary. It usually occurs when a person presumes to evaluate
matters beyond the limits of his personal training and competence.
Almost without exception that was the situation with a decided majority
of Seventh-day Adventist leaders with respect to doctrinal matters for nearly
a century after 1844. That explains the inability of many if not most of the
participants in the historic 1919 Bible conference to resolve the doctrinal
issues on its agenda. Adventist administrators untrained in reliable principles of
biblical exegesis have, almost without exception, nevertheless traditionally
functioned as the ultimate authority on matters of doctrine.

During the mid-century era (approximately 1940 to 1969) when, for the first
time, Adventist Bible scholars began to practice objective methods of Bible
study and church administrators, appreciating the value of their expertise,
began to accept them as genuine partners in dealing with doctrinal matters.
Biblical and doctrinal obscurantism gradually disappeared. After 1969,
however, as obscurantism on the part of new church administrators gave the
next decade (1969-1979) the unhappy sobriquet ``decade of obscurantism.''

For instance, during sessions of the Biblical Research Committee (now
Institute) Gerhard Hasel repeatedly stated that it was a mistake even to try
to be objective. In the plenary session of the Sanctuary Review Committee at
Glacier View, for instance, he demonstrated this by emphatically declaring
in the plenary session Monday afternoon, August 10, 1980, ``God's only
intention in Daniel 8:14 was to point forward to 1844!'' This statement was
met by a loud chorus of amens.

Obscurantism was also evident on the part of leaders in charge of study
Group 2 at Glacier View on Monday morning. Twelve of the sixteen speeches in
the group that morning favored Ford's point of view, but when chairman of
the group--a GC vice president--summed up the opinion of the group for its
report to the plenary session that afternoon, he reported the minority of
four speeches as the view of the majority--an obvious instance of
obscurantism. Following one of the speeches favoring Ford, the other vice
president present responded, ``We could never accept that!'' In the plenary
session that afternoon eleven of the fifteen speeches by Bible scholars
likewise favored Ford's position on the same topic, but again administration 
took the consensus to be negative. From beginning to end obscurantism was in
charge at Glacier View.

Obscurantism characterizes the tedious printed reports of the General
Conference-appointed Daniel and Revelation Committee that functioned during
the 1980s. (See below). It is likewise the guiding principle of the
Adventist Theological Society, legitimate heir of Gerhard Hasel's
hermeneutical legacy.

Obscurantism continues to be alive and well at the General Conference level.
On November 15, 2000 I sent another major paper on Daniel 8:14 to some
eighty Bible scholars and administrators, including the president of the
General Conference. His reply was courteous to a ``T'', but he referred the
paper to the Biblical Research Institute (BRI) with the comment that their
reply would be his also. In January 2001 he sent me a copy of the evasive
BRI reply, which reported that they had already considered and settled all
of the biblical anomalies in the traditional sanctuary doctrine to which my
paper had called attention, which I well knew was not so. Evidently
obscurantism is still in charge at BRI and the General Conference.

In what does official obscurantism with respect to the sanctuary doctrine
consist? Throughout the twentieth century, inclusive of Glacier View (1980)
and the subsequent Daniel and Revelation Committee Series report, the
General Conference has always countered flaws in the doctrine that have been
called to its attention with ever more elaborate and evasive reasons adduced
in favor of it. But it has never yet paid attention to the flaws themselves!

As long ago as 1934 W. W. Prescott called attention to this problem in a
letter he wrote to W. A. Spicer, president of the General Conference: ``I
have waited all these years for someone to make an adequate answer to
Ballenger, Fletcher and others on their positions re.\ the sanctuary but I
have not seen or heard it.''\sidenote{W. W. Prescott's letter is on file in 
the GC Archives.} Having been a member of the GC committees
that met with Ballenger, Fletcher, and Conradi, Prescott realized that the
official GC responses, both oral and published, offered presumed reasons for
believing the sanctuary doctrine, but left the flaws to which the three had
called attention completely unanswered! The same was true with respect to
questions regarding exegetical flaws in the sanctuary doctrine.
