\chapter{Flaws in the Traditional Sanctuary Doctrine}
\label{ch:flaws}

There can be no question as to the sincerity, diligence, and integrity of
those who formulated the traditional Adventist interpretation of
Daniel 8:14. It is equally obvious that they were following the flawed
principles of the prooftext method:
\begin{enumerate}
	\item In four major instances they adopted
translation errors where the KJV misrepresents the Hebrew text. 
	\item They
completely ignored the literary context in which Daniel 8:14 occurs.
	\item They likewise ignored the historical context specified by the first six
chapters and chapter 9:1--19 of the book, within which its several prophetic
pericopes were given and to which they specifically applied.
	\item They did not take into account the salvation history perspective 
specified by the 
        book (and the entire Old Testament),\sidenote[][-2cm]{\noteXXXVtext}~within which Daniel 8:14 occurs and
to which Daniel specifically applies it.
\end{enumerate}
As set forth in the preceding
chapter of this paper, sola Scriptura and the historical method both require
that these factors be taken into account.

Today, anyone who makes exegetical blunders such as these is automatically
dismissed as an unreliable Bible student. Had the pioneers of our message
been following the principles of the historical method they would never have
come to the conclusions they did---and never experienced the bitter
disappointment on October 22, 1844. Let us emulate their sincerity,
earnestness, and devotion to the Word of God, and be true to the best we
know today, as they were in their time!

In comparison with the exegetical requirements set forth in the two
preceding chapters (\nameref{ch:rightly} and \nameref{ch:explain} above), the 
traditional interpretation of Daniel 8:14 ignores
\begin{itemize}
	\item the historical context provided by chapters 1 to 6 and 9:4--19, within
which Inspiration placed it---the point in history when the seventy years of
exile foretold by Jeremiah came to a close and the restoration era was
about to begin.
	\item the salvation history perspective of Daniel's time, and of the entire
Bible.\sidenote{See previous note and~\nameref{ch:shp}}
	\item the Hebrew text of Daniel 8:14 and 9:25--26 at four major points,
identified in chapter \nameref{ch:explain} above.\sidenote{The prophetic 
day-for-a-literal-year concept was originally
formulated by the Karaite Jewish scholar Nahawendi in the ninth century in
an endeavor to identify events of his time as the fulfillment of Daniel's
prophecies. The idea that this ``principle'' was operative with respect to the
seventy ``weeks'' of years of Daniel 9 ignores the fact that it was, as a
matter of fact, an application of the ancient Jewish jubilee-year system of
dating, not the purported day-for-a-year ``principle.'' The ancient Jewish
Book of Jubilees uses this system of dating scores of times for dating
events in Jewish history. See Chapter 15, ``Jewish Interpretation of Daniel'',
in my Eschatology of Daniel for a number of relevant examples from the Book
of Jubilees. See also \bibentry{42b}, pp. 52--55, 208; 
\bibentry{102c}, p. 713; \bibentry{102d}, p. 196}
	\item the immediate context of 8:14 in chapter 8 itself, which explicitly
identifies
	\begin{enumerate}
		\item the sanctuary mentioned in verse 14 as that located by verses
9 to 11 in ``the beautiful land,'' Judea; 
		\item its desolation of the sanctuary
as that caused by the little horn in verses 11 to 13, and 
		\item when that
desolation would take place, at the close of the (Hellenistic) Greek era, in
verses 21 to 23.
	\end{enumerate}
	Accordingly, reference by analogy to the heavenly sanctuary
	of the Book of Hebrews is irrelevant.

	\item the fact that 9:24--26 has the sanctuary restored and in full operation
during the very time that 8:13--14 has it desolate and out of operation. This
contradiction, inherent in and essential to the traditional interpretation
of Daniel 8:14 which requires that the seventy weeks of years be considered
the first segment of the 2,300 ``days,'' renders it an exegetical oxymoron.
\end{itemize}

The day-for-a-year idea applied to Bible prophecy appears first in the ninth
century Karaite Jewish scholar Nahawendi's attempt to relate the fulfillment
of Daniel's prophecies to events of his day. Modern reliance on the
day-for-a-year ``principle'' in the interpretation of Bible prophecy
originated with
\begin{enumerate}
	\item the mistaken KJV rendition of the Hebrew \textit{erev boquer}
(``evenings mornings'') in Daniel 8:14 as ``days,'' when as a matter of fact 
\textit{erev boquer} is verse 14's contextual equivalent of ``regular burnt offering''
in the question of verse 13, to which verse 14 is the inspired answer, and
with 
	\item the endeavor to correlate these supposed ``days'' with the ``seventy
weeks'' of Daniel 9:24.
\end{enumerate}
The expression ``seventy weeks'' is simply use of the
jubilee system of expressing 490 years as 49 jubilees, each of its ten
``jubilees'' consisting of 49 literal years. There is absolutely no Bible
basis whatever for citing Daniel 9 as evidence for the day-for-a-year idea.

It should be noted that the ``days'' of Numbers 14:34 during which
representatives of the twelve tribes had spied out the land of Canaan were
not prophetic of the years God sentenced the Israelites to wander in the
desert. Those years were, rather, judicial, sentencing the unbelieving
wanderers for their lack of faith in God's promise to give them the land of
Canaan. The 390 ``days'' of Ezekiel 4:6 during which God directed the prophet
to lie on one side and then the other, represented that many past years of
apostasy. Those ``days'' were in no sense prophetic of the past years of
apostasy.

Under the caption ``Christ's Ministry in the Heavenly Sanctuary'' article 23
of Fundamental Beliefs reads as follows, with a distinction between that
which accurately reflects Scripture and is biblically relevant in bold face,
and the sanctuary doctrine's flawed interpretation of Bible passages in
ordinary type:

\begin{quote}
\textbf{There is a sanctuary in heaven, the true tabernacle which
the Lord set up and not man. In it Christ ministers in our behalf,
making
available to believers the benefits of His atoning sacrifice offered once
for all on the cross. He was inaugurated as our great High Priest and
began
His intercessory ministry at the time of His ascension.} In 1844, at the
end
of the prophetic period of 2300 days, He entered the second and last phase
of His atoning ministry. It is a work of investigative judgment which is
part of the ultimate disposition of all sin, typified by the cleansing of
the ancient sanctuary on the Day of Atonement. In that typical service the
sanctuary was cleansed with the blood of animal sacrifices, but the heavenly
things are purified with the perfect sacrifice of the blood of Jesus. The
investigative judgment reveals to heavenly intelligences who among the dead
are asleep in Christ and therefore, in Him, are deemed worthy to have part
in the first resurrection. It also makes manifest who, among the living, are
abiding in Christ, keeping the commandments of God and the faith of Jesus,
and in Him therefore, are ready for translation into His everlasting
kingdom. This judgment vindicates God in saving those who believe in Jesus.
It declares that those who have remained loyal to God shall receive the 
kingdom. The completion of this ministry of Christ will mark the close of
human probation before the second Advent.
\end{quote}

The first part of the preceding statement accurately reflects the
description of Christ's ministry on our behalf since His return to heaven
nearly two thousand years ago. The last part has no basis whatever in
Scripture. To be in harmony with the sola Scriptura principle it should be
deleted from the Fundamental Beliefs resume of Adventist beliefs and
replaced by an amplification of Christ's ministry as set forth in the Book
of Hebrews.

The ephemeral umbilical cord is essential to life prior to birth, but
totally irrelevant thereafter. May it be that the traditional sanctuary
doctrine was a sort of spiritual umbilical cord God permitted as a means of
reviving advent expectancy, but should be discarded once it had served its
purpose? ``The Son of Man is coming at an unexpected hour,'' ``the night is far
gone, the day is near,'' ``let us put on the armor of light.'' ``What sort of
persons ought you to be in leading lives of holiness and godliness'' while
``waiting for and hastening the day of God.''?\sidenote{\nameref{ch:matt24}:44;

Romans 13:12 the night is far gone, the day is near. Let us then lay aside the 
works of darkness and put on the armor of light;

2 Peter 3:11--12 Since all 
these things are to be dissolved in this way, what sort of persons ought you to 
be in leading lives of holiness and godliness, waiting for and hastening the 
coming of the day of God, because of which the heavens will be set ablaze and 
dissolved, and the elements will melt with fire?} May it be that God
overlooked this defect in their understanding of Daniel 8:14 and honored
their sincerity, in view of the fact that the traumatic experience of
October 22, 1844 had the effect of reviving the state of advent expectancy
Jesus long ago commended to His followers: ``Keep awake, therefore, for you
do not know on what day your Lord is coming.''\sidenote{\nameref{ch:matt24}:42}

The basic cause of the bitter disappointment was unawareness of the fact
that, when given, Daniel's preview of the future applied specifically to the
Jewish captives in Babylon anticipating return to their homeland, and to His
plans for them culminating in the establishment of His eternal reign of
righteousness in the long ago. This becomes obvious when the historical
circumstances of Daniel's time and its perspective of salvation
history---all explicit in the book itself---are taken into consideration.
The presupposition that Daniel 8:14, when given, anticipated events of our
time was the basic cause of the 1844 error and the resulting disappointment.
Continued disappointment will be inevitable until this error is recognized
and corrected, and the historicist principle on which it is based, is
abandoned.
